\section{第二章 Poisson过程}
\begin{problem}{2.1}
$N(t)$为一Possion过程, 对$s<t$试求条件概率$\p\{N(s) = k | N(t) = n\}$.
\end{problem}
\begin{solution}
	\[\begin{aligned}
			  & \p\{N(s)=k|N(t)=n\} = \frac{\p[N(s)=k, N(t)=n]}{\p[N(t)=n]}                                                                                                                                   \\
			= & \frac{\p[N(s)-N(0)=k, N(t)-N(s)=n-k]}{\p[N(t)=n]}                                                                                                                                             \\
			= & \left[\frac{({\lambda s})^k\e^{-{\lambda s}}}{k!} \cdot \frac{\left(\lambda(t-s)\right)^{n-k} \e^{-\lambda(t-s)}}{(n-k)!}\right] \Bigg/ \left[\frac{(\lambda t)^n \e^{-\lambda t}}{n!}\right] \\
			= & \frac{n!}{k!(n-k)!}\left(\frac{s}{t}\right)^k \left(1-\frac{s}{t}\right)^{n-k}                                                                                                                \\
		\end{aligned}\]
\end{solution}

\begin{problem}{2.2}
$\{N(t), t \geqslant 0\}$为一强度是$\lambda$的Possion过程. 对$s > 0$试计算$E[N(t) \cdot N(t+s)]$.
\end{problem}
\begin{solution}
	\[\begin{aligned}
			\text{原式} & = \E\Big\{ N(t)\big[ N(t+s)-N(t)+N(t)\big] \Big\}       \\
			          & = \E\Big\{ N(t)\big[ N(t+s)-N(t)\big] \Big\}+\E[N^2(t)] \\
			          & = \lambda t \cdot \lambda s + \var[N(t)] + \E^2[N(t)]   \\
			          & = \lambda ^2 ts+\lambda t + (\lambda t)^2               \\
			          & = \lambda ^2 t(s+t)+\lambda t
		\end{aligned}\]
\end{solution}

\begin{problem}{2.3}
电报依平均速率为每小时3个的Possion过程到达电报局, 试问:
\begin{enumerate}[label=(\roman*)]
	\item 从早上八时到中午没收到电报的概率;
	\item 下午第一份电报到达时间的分布是什么?
\end{enumerate}
\end{problem}
\begin{solution}
	\begin{enumerate}[label=(\roman*)]
		\item 令$t$的计时单位为小时, 并以早上8:00为起始时刻,所求事件的概率即
		      \[\p[N(4)=0]=\frac{\e^{-3\times 4} \cdot (3\times 4)^0}{0!} = \frac{1}{\e^{12}} \approx 6.1 \times 10^{-6}\]
		\item 取中午12:00为起始时刻, $T$表示下午第一份电报到达时间
		      \[F(t) = \p(T \leqslant t) = \p[N(t) \geqslant 1] = 1 - \e^{-3t}\]
		      $\displaystyle \therefore f(t) = \frac{\d F_T (t)}{\d t} = 3\e^{-3t}$, 即$T$服从参数为3的指数分布\\
	\end{enumerate}
\end{solution}

\begin{problem}{2.4}
$\{N(t), t \geqslant 0\}$为一$\lambda = 2$的Possion过程, 试求:
\begin{enumerate}[label=(\roman*)]
	\item $P\{N(1) \leqslant 2\}$;
	\item $P\{N(1) = 1 \text{\,且\,}N(2) = 3\}$;
	\item $P\{N(1) \geqslant 2 | N(1) \geqslant 1\}$.
\end{enumerate}
\end{problem}
\begin{solution}
	\begin{enumerate}[label=(\roman*)]
		\item $\displaystyle \p[N(1)\leqslant 2] = \sum_{k=0}^{2}\p[N(1)=k]=\sum_{k=0}^{2}\frac{\e^{-2}2^k}{k!} = \frac{5}{\e^2}$
		\item $\displaystyle \text{原式} = \p[N(1)-N(0)=1]\p[N(2)-N(1)=2] = \frac{\e^{-2}2}{1!}\cdot \frac{\e^{-2} 2^2}{2!} = \frac{4}{\e^4}$
		\item \[\begin{aligned}
				      \text{原式} & = \frac{\p[N(1) \geqslant 2 | N(1) \geqslant 1]}{\p[N(1) \geqslant 1]} = \frac{\p[N(1) \geqslant 2]}{\p[N(1) \geqslant 1]} = \frac{1-\p[N(1)-N(0) \leqslant 1]}{1-\p[N(1)-N(0)=0]} \\
				                & = \frac{1-\frac{\e^{-2}2^0}{0!}-\frac{\e^{-2}2^1}{1!}}{1-\frac{\e^{-2}2^0}{0!}} = \frac{1-3\e^{-2}}{1-\e^{-2}} = \frac{\e^{2}-3}{\e^{2}-1}
			      \end{aligned}\]
	\end{enumerate}
\end{solution}

\begin{problem}{2.5}
证明概率$P_m(t) = \p[N(t) = m]$在命题2.1的假定$(1)\sim(4)$下满足微分方程
\[P'_m(t) = -\lambda P_m(t) + \lambda P_{m-1}(t),\quad m=1,2,\cdots, \]
并证明在初始条件下$P_m(0) = 0, m = 1,2,\cdots$下的解为$\displaystyle \frac{{\lambda}^m t^m}{m!}\e^{-\lambda t}$.
\par 4个假定分别为:
\begin{enumerate}[label=(\arabic*)]
	\item 在不相交区间中事件发生的数目相互独立,也即对任何整数$n=1,2,\cdots $,设时刻$t_0=0<t_1<t_2<\cdots <t_n$,增量$N(t_1)-N(t_0),N(t_2)-N(t_1),\cdots ,N(t_n)-N(t_{n-1})$相互独立;
	\item 对任何时刻$t$和正数$h$,随机变量(增量)$N(t+h)-N(t)$的分布只依赖于区间长度$h$而不依赖于时刻$t$;
	\item 存在正常数$\lambda $,当$h\downarrow 0$时,使在长度为$h$的小区间中事件至少发生一次的概率\[\p[N(t+h)-N(t)\geqslant 1]=\lambda h+o(h);\]
	\item 在小区间$(t,t+h]$发生两个或以上事件的概率为$o(h)$(可以忽略不计),即当$h\downarrow 0$,\[\p[N(t+h)-N(t)\geqslant 2]=o(h)\]
\end{enumerate}
\end{problem}
\begin{solution}
	本题暂缺
\end{solution}

\begin{problem}{2.6}
一部600页的著作总共有240个印刷错误, 试利用Possion过程近似求出某连续三页无错误的概率.
\end{problem}
\begin{solution}
	设Poisson参数为$\lambda $,有$600\lambda = 240 \Rightarrow \lambda = 0.4$
	\[\p[N(m+3)-N(m)=0]=\frac{\e^{-0.4\times 3}(0.4\times 3)^0}{0!} = \frac{1}{\e^{1.2}}\]
\end{solution}

\begin{problem}{2.7}
$N(t)$是强度为$\lambda$的Possion过程. 给定$N(t) = n$, 试求第$r$个事件$(r \leqslant n)$发生的时刻$W_r$的条件概率密度$f_{W_r|N(t)=n}(w_r|n)$.
\end{problem}
\begin{solution}[1]
	\[\begin{aligned}
			  & f_{W_r|N(t)=n}(w_r|n) \cdot \Delta w_r                                                                                                                                                                                                 \\
			= & \p[N(w_r) - N(0) = r - 1, N(w_r+\Delta w_r)-N(w_r)=1|N(t) = n]                                                                                                                                                                         \\
			= & \p[N(w_r) - N(0) = r - 1] \cdot \p[N(w_r+\Delta w_r)-N(w_r)=1]                                                                                                                                                                         \\
			  & \cdot \frac{\p[N(t)-N(w_r+\Delta w_r)=n-r]}{\p[N(t) = n]}                                                                                                                                                                              \\
			= & \frac{\frac{(\lambda w_r)^{r-1}}{(r-1)!}\e^{-\lambda w_r} \cdot [\lambda \Delta w_r + o(\Delta w_r)] \cdot \frac{[\lambda (t-w_r-\Delta w_r)]^{n-r}}{(n-r)!} \e^{-\lambda(t-w_r-\Delta w_r)}}{\frac{(\lambda t)^n}{n!} \e^{\lambda t}} \\
		\end{aligned}\]
	两边除以$\Delta w_r$并令$\Delta w_r\to 0$得
	\[f_{W_r|N(t)=n}(w_r|n) = \frac{n!}{(r-1)!(n-r)!}\frac{(w_r)^{r-1}(t-w_r)^{n-r}}{t^n}\]
\end{solution}
\begin{solution}[2]
	直观理解如下
	\begin{figure}[H]
		\centering
		\begin{tikzpicture}[>=Stealth]
			\draw[->](-1,0)--(8,0)node[below right]{$T$};
			\fill (0,0)circle(1pt);
			\draw (0,-0.13)--(0,0)node[below left]{$0$};
			\draw[->] (1,0.35)node[above]{$1_{st}$}--(1,0);
			\draw[->] (2,0.35)node[above]{$2_{nd}$}--(2,0);
			\draw (2.5,0.35)node[]{$\cdots$};
			\draw[->](3,0.35)node[above]{$(r\!-\!1)_{th}$}--(3,0);
			\draw[->](4,0.35)node[above]{$r_{th}$}--(4,0)node[below]{$w_r$};
			\draw (4.5,0.35)node[]{$\cdots$};
			\draw[->](6,0.35)node[above]{$n_{th}$}--(6,0)node[below right]{$t$};
			\draw[decorate,decoration={brace,amplitude=3mm}] (4,-0.15)--(0,-0.15);
			\draw[decorate,decoration={brace,amplitude=3mm}] (6,-0.15)--(4,-0.15);
		\end{tikzpicture}
	\end{figure}
	将之分为三段,然后得到类似多项分布的结论(分三组乘起来)
	\[f_{W_r|N(t)=n}(w_r|n) = \frac{n!}{(r-1)!\cdot 1 \cdot (n-r)!}\left(\frac{w_r}{t}\right)^{r-1}\cdot \frac{1}{t}\cdot \left(\frac{t-w_r}{t}\right)^{n-r}\]
\end{solution}
\begin{solution}[3]
	比较暴力的方法.因为泊松分布两次事件之间的时间间隔$W_{i+1}-W_{i}=\delta_i$遵循指数分布$\delta_i\sim \Exp(\lambda)$,利用$\mathbf{1.13}$的结论$W_r=\sum_{i=1}^{r}\delta_i \sim \Gamma(r,\lambda)$有:
	\[\begin{aligned}
			f_{W_r}(w_r)          & = \frac{\lambda \e^{-\lambda w_r}(\lambda w_r)^{r-1}}{(r-1)!}                                                                                                                \\
			f_{W_r|N(t)=n}(w_r|n) & = \frac{f_{W_r}(w_r)\cdot \p(\text{$t-w_r$间到达了$n-r$次})}{\p[(N(t)=n]} = \frac{f_{W_r}(w_r)\cdot \p[N(t)-N(w_r)=n-r]}{\p[N(t)=n]}                                              \\
			                      & = \frac{\frac{\lambda \e^{-\lambda w_r}(\lambda w_r)^{r-1}}{(r-1)!}\cdot \frac{\e^{-\lambda(t-w_r)}[\lambda(t-w_r)]^{n-r}}{(n-r)!}}{\frac{\e^{-\lambda t}(\lambda t)^n}{n!}} \\
			                      & = \frac{n!}{(r-1)!(n-r)!}\frac{(w_r)^{r-1}(t-w_r)^{n-r}}{t^n}
		\end{aligned}\]
\end{solution}

\begin{problem}{2.8}
令$\{N_i(t), t\geqslant 0\}, i = 1,2,\cdots, n$为n个独立的有相同强度参数$\lambda$的Possion过程. 记$T$为在全部$n$个过程中至少发生了一件事的时刻, 试求$T$的分布.
\end{problem}
\begin{solution}[1]
	本题与1.14几乎一样,是Poisson和 \Exp 的一体两面.\\
	记$N(t)=N_1(t)+\cdots +N_n(t)\Rightarrow N(t)\sim \poi (n\lambda t)$,其对应首达时$X\sim \Exp(n\lambda)$
	\begin{align*}
		F_T(t) & = \p[T \leqslant t] = 1 - \p(T<t)        \\
		       & = 1 - \p[N_i(t) = 0, i= 1, 2, \cdots, n] \\
		       & = 1 - \left(\e^{-\lambda t}\right)^n
	\end{align*}
	$\therefore f_T(t) = F'_T(t) = n\lambda \e^{-n\lambda t}$\\
	即$T$服从参数为$n\lambda$的指数分布.
\end{solution}
\begin{solution}[2]
	大差不差,$N(t)=\sum_{i=1}^{n}N_i(t)\sim \poi(n\lambda t)$,那么
	\[\begin{aligned}
			\p[N(t)\geqslant 1] & = \sum_{k=1}^{\infty}\p[N(t)=k] = 1-\p[N(t)=0]                   \\
			                    & = 1-\e^{-n\lambda t}                                             \\
			f_T(t)              & = \frac{\d}{\d t}\p[N(t)\geqslant 1] = n\lambda \e^{-n\lambda t}
		\end{aligned}\]
\end{solution}

\begin{problem}{2.9}
考虑参数为$\lambda$的Possion过程$N(t)$, 若每一事件独立地以概率$p$被观察到, 并将观察到的过程记为$N_1(t)$. 试问$N_1(t)$是什么过程? $N(t)-N_1(t)$呢? $N_1(t)$与$N(t) - N_1(t)$是否独立?
\end{problem}
\begin{solution}
	由题设得
	\begin{enumerate}[label=(\roman*)]
		\item $N_1(0)=0$
		\item $\{N_1(t):t\geqslant 0\}$是独立增量过程
	\end{enumerate}
	从而,对$0\leqslant s<t$,$N_1(t)-N_1(s)$服从参数为$\lambda p(t-s)$的Poisson分布.
	故$\{N_1(t):t\geqslant 0\}$是参数为$\lambda p$的Possion过程.
\end{solution}

\begin{problem}{2.10}
到达某加油站的公路上的卡车服从参数为${\lambda}_1$的Possion过程$N_1(t)$, 而到达的小汽车服从参数为${\lambda}_2$的Possion过程$N_2(t)$, 且过程$N_1(t)$与$N_2(t)$独立. 试问随机过程$N(t) = N_1(t) + N_2(t)$是什么过程? 并计算在总车流数$N(t)$中卡车首先到达的概率.
\end{problem}
\begin{solution}
	\[g_N(v)  = g_{N_1}(v) \cdot g_{N_2}(v) = \e^{\lambda_1 v(\e^t-1)}\e^{\lambda_2 v(\e^t-1)} = \e^{(\lambda_1 +\lambda_2)\v(\e^t-1)}\]
	$\therefore N(t)$是强度为$\lambda_1+\lambda_2$的Possion过程\\
	记$W_1,W_2$分别为卡车、小汽车的第一次到达时间, 则$W_1$服从参数为$\lambda_1$的指数分布, $W_1$服从参数为$\lambda_1$的指数分布.
	\[\begin{split}
			\therefore \p(W_1 < W_2) & = \iint\limits_{0 \leqslant W_1 < W_2 \leqslant +\infty}f_{W_1,W_2}(w_1,w_2)\d w_1\d w_2\\
			& = \int^{+\infty}_0\d w_1\int^{+\infty}_{w_1}\lambda_1\lambda_2\e^{-w_1\lambda_1-w_2\lambda_2}\,\d w_2\\
			& = \frac{\lambda_1}{\lambda_1+\lambda_2}
		\end{split}\]
	或者另一个思路:卡车的首达时$T\sim \Exp(\lambda_1)$,在$(0,T]$之间没有汽车到达
	\[\begin{aligned}
			\p[T=t,N_2(t)=0] & = (\lambda_1 \e^{-\lambda_1 t}\d t)\cdot \frac{\e^{-\lambda_2 t}(\lambda_2 t)^0}{0!} = \lambda_1 \e^{-(\lambda_1+\lambda_2)t}\d t \\
			\p(W_1<W_2)      & = \int_{0}^{+\infty}\p[T=t,N_2(t)=0] = \frac{\lambda_1}{\lambda_1+\lambda_2}
		\end{aligned}\]
\end{solution}

\begin{problem}{2.11}
冲击模型(Shock Model)记$N(t)$为某系统到某时刻$t$受到的冲击次数, 它是参数为$\lambda$的Possion过程. 设第k次冲击对系统的损害大小$Y_k$服从参数为$\mu$的指数分布, $Y_k, k = 1, 2, \cdots, $独立同分布. 记$X(t)$为系统所受到的总损害. 当损害超过一定的极限$\alpha$时系统不能运行, 寿命终止, 记$T$为系统寿命. 试求该系统的平均寿命$\E(T)$, 并对所得结果作出直观解释.
\par 提示:对非负随机变量$\displaystyle \E(T) = \int^{\infty}_0 \p(T>t)\d t$\\
\end{problem}
\begin{solution}[1]
	\[\begin{split}
			\p(T>t) & = \p\{X(t) \leqslant \alpha\} = \p\left\{\sum^{N(t)}_{k=1}Y_k \leqslant \alpha \right\}\\
			& = \sum^{\infty}_{n=0}\p\Bigg\{\sum^{N(t)}_{k=1}Y_k \leqslant \alpha \Bigg| N(t) = n\Bigg\} \cdot \p\{N(t) = n\}\\
			& = \sum^{\infty}_{n=0}\p\Bigg\{\sum^n_{k=1}Y_k \leqslant \alpha \Bigg| N(t) = n\Bigg\} \cdot \p\{N(t) = n\}\\
			& = \sum^{\infty}_{n=0}\p\{W_n \leqslant \alpha\} \cdot \p\{N(t) = n\}\\
		\end{split}\]
	求和式中当$n=0$时认为$\p\{W_n \leqslant \alpha | N(t) = n\} = 1$\\
	$\because Y_k \sim \Exp(\mu),\quad \therefore W_n = \sum\limits^n_{k=1}Y_k \sim \Gamma(n,\mu)$
	\[\begin{split}
			\p(W_n \leqslant \alpha) & = \frac{\mu^n}{(n-1)!}\int^\alpha_0s^{n-1}\e^{-\mu s}\d s\quad (n\geqslant 1)\\
			\p[N(t) = n] & = \frac{(\lambda t)^n}{n!}\e^{-\lambda t}
		\end{split}\]
	\[\therefore \p(T>t) = \e^{-\lambda t} + \e^{-\lambda t}\sum^{\infty}_{n=1}\frac{(\lambda\mu t)^n}{n!(n-1)!}\int^\alpha_0 s^{n-1}\e^{-\mu s}\d s\]
	\[\begin{split}
			\therefore \E(T) & = \int^{+\infty}_0 \p(T>t)\d t\\
			& = \frac{1}{\lambda} + \sum^{\infty}_{n=1}\frac{(\lambda\mu)^n}{n!(n-1)!}\int^{\infty}_0 t^n \e^{-\lambda t}\d t\int^\alpha_0 s^{n-1}\e^{-\mu s}\d s\\
			& = \frac{1}{\lambda} + \frac{1}{\lambda}\sum^{\infty}_{n=1}\frac{\mu^n\Gamma(n+1)}{n!(n-1)!}\int^\alpha_0 s^{n-1}\e^{-\mu s}\d s\\
			& = \frac{1}{\lambda} + \frac{1}{\lambda}\int^\alpha_0 \bigg[\sum^{\infty}_{n=1}\frac{(\mu s)^{n-1}\e^{-\mu s}}{(n-1)!}\bigg]\d (\mu s)\\
			& = \frac{1+\mu\alpha}{\lambda}
		\end{split}\]
	从结果看, 若$\lambda$越大(系统所受冲击越频繁), $\mu$越小(每次冲击所造成的平均损害越大), $\alpha$越小(系统所能承受的的损害极限越小), 则系统平均寿命越短, 且当$\alpha$等于$0$时系统的平均寿命即为第一次冲击到来的平均时间, 符合常识.
\end{solution}
\begin{solution}[2]
	\[G_n(\alpha) = \p\{Y_1+\cdots+Y_k \leqslant \alpha\} = \p\{W_n \leqslant \alpha\}= \p\{N_1(\alpha) \geqslant n\} = \sum^{\infty}_{k=n}\frac{(\mu\alpha)^n}{k!}\e^{-\mu\alpha}\]
	其中$N_1(t)$是强度为$\mu$的Possion过程\\
	\[\begin{split}
			\therefore \p(T>t) & = \p\{X(t) \leqslant \alpha\}\\
			& = \sum^{\infty}_{n=0}\frac{(\lambda t)^n \e^{-\lambda t}}{n!}G_n(\alpha) \qquad (\text{课本$\mathbf{P}_{21}$例2.4})\\
			& = \sum^{\infty}_{n=0}\sum^{\infty}_{k=n}\frac{(\lambda t)^n}{n!}\e^{-\lambda t}\frac{(\mu\alpha)^n}{k!}\e^{-\mu\alpha}\\
			& = \sum^{\infty}_{k=0}\frac{(\mu\alpha)^n}{k!}\e^{-\mu\alpha}\sum^{k}_{n=0}\frac{(\lambda t)^n}{n!}\e^{-\lambda t}
		\end{split}\]
	\[\begin{split}
			\therefore \E(T) & = \int^{+\infty}_0 \p(T>t)\d t\\
			& = \sum^{\infty}_{k=0}\frac{(\mu\alpha)^k}{k!}\e^{-\mu\alpha}\sum^{k}_{n=0}\int^{+\infty}_0\frac{(\lambda t)^n}{n!}\e^{-\lambda t}\d t\\
			& = \sum^{\infty}_{k=0}\frac{(\mu\alpha)^k}{k!}\e^{-\mu\alpha}\frac{(k+1)\Gamma(n+1)}{n!\lambda}\\
			& = \frac{1}{\lambda}\sum^{+\infty}_{k=0}\frac{(\mu\alpha)^k}{k!}\e^{-\mu\alpha} + \frac{\mu\alpha}{\lambda}\sum^{+\infty}_{k=1}\frac{(\mu\alpha)^{k-1}}{(k-1)!}\e^{-\mu\alpha}\\
			& = \frac{1+\mu\alpha}{\lambda}
		\end{split}\]
\end{solution}
\begin{solution}[3\footnote{Sol \textcolor{red}{3} 和Sol \textcolor{red}{4} 都是我写的,计算结果为$\frac{\mu \alpha}{\lambda}$.但实际结果应为$\frac{1+\mu \alpha}{\lambda}$,懒得去找漏项了}]
	\[\begin{aligned}
			1-F_T(t) & =\p(T<t)=\p[X(t)<\alpha ]=\p\left(\sum_{k=1}^{N(t)}Y_k\leq \alpha \right)                    \\
			         & = \sum_{n=1}^{\infty}\p \left(\sum_{k=1}^{N(t)}Y_k\leq \alpha \Bigg|N(t)=n\right)\p [N(t)=n] \\
			         & = \sum_{n=1}^{\infty}\p \left(\sum_{k=1}^{n}Y_k\leq \alpha \right)\p [N(t)=n]
		\end{aligned}\]
	$Y_1,Y_2,\dots ,Y_n\iid$且$Y_i\sim \Exp(\mu )$.设$S_n=\sum_{k=1}^{n}Y_k$
	由独立同分布的指数分布随机和为参数为$(n,\mu)$的$\Gamma$分布
	\[f_{S_n}(s)=\mu \frac{\e ^{-\mu s}(\mu s)^{n-1}}{(n-1)!}\]
	\[\p(S_n\leq \alpha)=\int_{0}^{\alpha}\mu \frac{\e ^{-\mu s}(\mu s)^{n-1}}{(n-1)!}\d s=\frac{\mu}{(n-1)!}\int_{0}^{\alpha}\e^{-\mu s}(\mu s)^{n-1}\d s\]
	\[\begin{aligned}
			1-F_T(t) & =\sum_{n=1}^{\infty}\frac{\e ^{-\lambda t}(\lambda t)^n}{n!}\cdot \frac{\mu}{(n-1)!}\int_{0}^{\alpha}\e^{-\mu s}(\mu s)^{n-1}\d s                                        \\
			         & =\mu \e^{-\lambda t}\sum_{n=1}^{\infty}\int_{0}^{\alpha}\frac{(\lambda t)^n}{n!(n-1)!}\e^{-\mu s}(\mu s)^{n-1}\d s                                                       \\
			         & \xlongequal{\text{\textcolor{red}{交换求和和积分符号}}}\mu \e^{-\lambda t}\int_{0}^{\alpha}\sum_{n=0}^{\infty}\frac{(\lambda t)^{n+1}}{(n+1)!}\frac{(\mu s)^n\e^{-\mu s}}{n!}\d s \\
		\end{aligned}\]
	因为$\displaystyle \sum_{n=0}^{\infty}\frac{(\lambda t)^{n+1}}{(n+1)!}=\e^{\lambda t}-1 ,\quad \sum_{n=0}^{\infty}\frac{(\mu s)^n \e^{-\mu s}}{n!}=1$
	\\ 所以由\textcolor{red}{Mertens定理}得$\displaystyle \sum_{n=0}^{\infty}\frac{(\lambda t)^{n+1}}{(n+1)!}\frac{(\mu s)^n \e^{-\mu s}}{n!}=(\e^{\lambda t}-1)\cdot 1=\e^{\lambda t}-1$
	\[\begin{aligned}
			1-F_T(t) & =\mu \e^{-\lambda t}\int_{0}^{\alpha}(\e^{\lambda t}-1)\cdot 1\d s=\mu \alpha (1-\e^{-\lambda t}) \\
			f_T(t)   & =-\frac{\d}{\d t}(1-F_T(t))=\lambda \mu \alpha \e^{-\lambda t}                                    \\
			\E(T)    & =\int_{0}^{+\infty}t f_T(t)\d t=\frac{\mu \alpha}{\lambda }
		\end{aligned}\]
\end{solution}
\begin{solution}[4\footnote{请查看上页的脚注}]
	可以采用不那么暴力的方法,即使用题目里的提示$\displaystyle \E(T)=\int_{0}^{+\infty}\p(T>t)\d t$,
	先把这个提示证一遍
	\[\begin{aligned}
			\int_{0}^{+\infty}\p(T>t)\d t & =\int_{0}^{+\infty}\left(\int_{t}^{+\infty}f_T(x)\d x\right)\d t\xlongequal{\text{交换积分次序}}                 \\
			                              & =\int_{0}^{+\infty}\d x \int_{0}^{x}f_T(x)\d t=\int_{0}^{+\infty}xf_T(x)\d x=\int_{0}^{+\infty}tf_T(t)\d t \\
			                              & =\E(T)
		\end{aligned}\]
	\[\begin{aligned}
			\p(T>t) & =1-F_T(t)=\mu \e^{-\lambda t}\int_{0}^{\alpha}\sum_{n=0}^{\infty}\frac{(\lambda t)^{n+1}}{(n+1)!}\frac{(\mu s)^n \e^{-\mu s}}{n!}\d s      \\
			\E(T)   & =\int_{0}^{+\infty}\p(T>t)\d t                                                                                                             \\
			        & =\sum_{n=1}^{\infty}\frac{(\mu \lambda)^n}{n!(n-1)!}\int_{0}^{+\infty}t^n \e^{-\lambda t}\d t\cdot \int_{0}^{\alpha}s^{n-1}\e^{-\mu s}\d s \\
			        & =\frac{1}{\lambda}\sum_{n=1}^{\infty}\frac{\mu ^n}{(n-1)!}\int_{0}^{\alpha}s^{n-1}\e^{-\mu s}\d s                                          \\
			        & =\frac{1}{\lambda}\int_{0}^{\alpha}\e^{-\mu s}\sum_{n=1}^{\infty}\frac{\mu ^n}{(n-1)!}s^{n-1}\d s                                          \\
			        & =\frac{\mu}{\lambda}\int_{0}^{\alpha}\d s=\frac{\mu \alpha}{\lambda}
		\end{aligned}\]
\end{solution}

\begin{problem}{2.12}
令$N(t)$是强度函数为$\lambda (t)$的非齐次Possion过程, $X_1, X_2, \cdots$为事件间的时间间隔.
\begin{enumerate}[label=(\roman*)]
	\item $X_i$是否独立;
	\item $X_i$是否同分布;
	\item 试求$X_1$与$X_2$的分布.
\end{enumerate}
\end{problem}
\begin{solution}[1]
	记$\displaystyle m(t) = \int^t_0\lambda(u)\d u$
	等待时间$W_1,W_2$的联合分布函数为
	\[\begin{split}
			F_{W_1,W_2}(t_1,t_2) & = \p(W_1 \leqslant t_1, W_2 \leqslant t_2),\qquad (0\leqslant t_1 < t_2)\\
			& = \p(N(t_1) \geqslant 1, N(t_2) \geqslant 2)\\
			& = \sum^{\infty}_{k=2}\sum^k_{\ell=1}\p[N(t_1) = \ell, N(t_2) = k]\\
			& = \sum^{\infty}_{k=2}\sum^k_{\ell=1}\p[N(t_1) = \ell, N(t_2) - N(t_1) = k - \ell]\\
			& = \sum^{\infty}_{k=2}\sum^k_{\ell=1}\p[N(t_1) = \ell)\p(N(t_2) - N(t_1) = k - \ell]\\
			& = \sum^{\infty}_{k=2}\sum^k_{\ell=1}\frac{(m(t_1)^\ell)}{\ell !}\e^{-m(t_1)}\cdot \frac{[m(t_2)-m(t_1)]^{k-\ell}}{(k-\ell)!}\e^{-[m(t_2)-m(t_1)]}\\
			& = \sum^{\infty}_{k=2}\sum^k_{\ell=1}\frac{(m(t_1))^\ell[m(t_2)-m(t_1)]^{k-\ell}}{\ell !(k-\ell)!}\e^{-m(t_2)}\\
			& = \e^{-m(t_2)}\sum^{\infty}_{k=2}\frac{1}{k!}\sum^k_{\ell=1}\binom{k}{\ell}[m(t_1)]^\ell[m(t_2)-m(t_1)]^{k-\ell}\\
			& = \e^{-m(t_2)}\sum^{\infty}_{k=2}\frac{1}{k!}\bigg\{\sum^k_{\ell=0}\binom{k}{\ell}[m(t_1)]^\ell[m(t_2)-m(t_1)]^{k-\ell}-[m(t_2)-m(t_1)]^k\bigg\}\\
			& = \e^{-m(t_2)}\Big\{\e^{m(t_2)}-\e^{m(t_2)-m(t_1)}-m(t_1)\Big\}\\
			& = 1-\e^{-m(t_1)}-m(t_1)\e^{-m(t_2)}\\
		\end{split}\]
	\[\therefore f_{W_1,W_2}(t_1,t_2) = \frac{\partial^2F_{W_1,W_2}(t_1,t_2)}{\partial t_1\partial t_2} = \lambda(t_1)\lambda(t_2)\e^{-m(t_2)}\]
	\[\because
		\begin{cases}
			W_1 = X_1 \\
			W_2 = X_1 + X_2
		\end{cases}\]
	$\therefore f_{X_2,X_2}(t_1,t_2) = \lambda(t_1)\lambda(t_1+t_2)\e^{-m(t_1+t_2)}$不能写为$g_1(t_1)g_2(t_2)$形式\\
	$\therefore X_1, X_2$不独立,又有
	\[f_{X_1}(t_1) = \lambda(t_1)\int^{+\infty}_0\lambda(t_1+t_2)\e^{-m(t_1+t_2)}\d t_2 = \lambda(t_1)\Big[\e^{-m(t_1)}-\e^{-m(+\infty)}\Big]\quad (t_1 > 0)\]
	下面确定$\e^{-m(\infty)}$:
	\[1 = \int^{+\infty}_0f_{X_1}(t_1)\d t_1 = \int^{+\infty}_0\lambda(t_1)\Big[\e^{-m(t_1)}-\e^{-m(+\infty)}\Big]\d t_1 = 1 - [m(+\infty)+1]\e^{-m(+\infty)}\]
	$\therefore \e^{-m(+\infty)} = 0$
	\[\begin{split}
			\therefore f_{X_1}(t_1) & = \lambda(t_1)\e^{-m(t_1)}\qquad (t_1>0)\\
			f_{X_2}(t_2) & = \int^{+\infty}_0\lambda(t_1)\lambda(t_1+t_2)\e^{-m(t_1+t_2)}\d t_1 \qquad (t_2>0)\\
		\end{split}\]
	$\therefore X_1,X_2$不同分布且其概率密度函数如上.
\end{solution}

\begin{problem}{2.13}
考虑对所有$t$, 强度函数$\lambda (t)$均大于$0$的非齐次Possion过程$\{N(t), t \geqslant 0\}$. 令$\displaystyle m(t) = \int^t_0 \lambda(u)\d u, m(t)$的反函数为$\ell(t)$, 记$N_1(t)= N(\ell(t))$. 试证$N_1(t)$是通常的Possion过程, 试求$N_1(t)$的强度参数$\lambda$.
\end{problem}
\begin{solution}
	\begin{enumerate}[label=(\roman*)]
		\item $N_1(0)=N(\ell(0))=N(0)=0$
		\item $\because m(\ell)$单增, $\therefore \ell(t)$单增\\
		      $\therefore $对任意$0 \leqslant t_1 < t_2 \cdots < t_n$, 有$0 \leqslant \ell(t_1) < \ell(t_2) < \cdots < \ell(t_n)$,且有
		      \begin{align*}
			      N_1(t_2) - N_1(t_1)     & = N(\ell(t_2)) - N(\ell(t_1))     \\
			      N_1(t_3) - N_1(t_2)     & = N(\ell(t_3)) - N(\ell(t_2))     \\
			      \vdots                                                      \\
			      N_1(t_n) - N_1(t_{n-1}) & = N(\ell(t_n)) - N(\ell(t_{n-1}))
		      \end{align*}
		      $\because N(t)$是独立增量过程 \quad $\therefore N_1(t)$也是独立增量过程
		\item $\forall \,0 \leqslant s < t$, 有
		      \[\begin{split}
				      \p(N_1(t) - N_1(s) = k) & = \p\{N(\ell(t)) - N(\ell(s)) = k\}\\
				      & = \frac{[m(\ell(t)) - m(\ell(s))]^k}{k!} \e^{-[m(\ell(t)) - m(\ell(s))]}\\
				      & = \frac{(t-s)^k}{k!}\e^{-(t-s)} \qquad (k=0,1,\cdots)
			      \end{split}\]
		      $\therefore N_1(t)$是强度为$1$的Possion过程
	\end{enumerate}
\end{solution}

\begin{problem}{2.14}
设$N(t)$为更新过程, 试判断下述命题的真伪:
\begin{enumerate}[label=(\roman*)]
	\item $\{N(t) < k \} \Longleftrightarrow \{W_k > t\}$;
	\item $\{N(t) \leqslant k \} \Longleftrightarrow \{W_k \geqslant t\}$;
	\item $\{N(t) > k \} \Longleftrightarrow \{W_k < t\}$;
\end{enumerate}
其中$W_k$为第$k$个事件的等待时间.
\end{problem}
\begin{solution}
	\begin{enumerate}[label=(\roman*)]
		\item $\displaystyle \{N(t) < k \} = \overline{\{N(t) \geqslant k\}} = \overline{\{W_k \leqslant t\}} = \{W_k > t\}$
		\item \[\begin{split}
				      \{N(t) \leqslant k \} & = \{N(t) < k + 1\} = \overline{\{N(t) \geqslant k + 1\}} = \overline{\{W_{k+1} \leqslant t\}}\\
				      & = \{W_{k+1} > t\} \neq \{W_k \geqslant t\}
			      \end{split}\]
		      \begin{figure}[H]
			      \centering
			      \begin{tikzpicture}[>=Stealth]
				      \draw[->](-0.5,0)--(0,0)node[below]{0}--(6,0)node[below right]{$T$};
				      \fill (0,0)circle(1pt);
				      \draw[->] (1,0.35)node[above]{$1_{st}$}--(1,0);
				      \draw[->] (2,0.35)node[above]{$2_{nd}$}--(2,0);
				      \draw (2.5,0.35)node[]{$\cdots$};
				      \draw[->](3,0.35)node[above]{$k_{th}$}--(3,0);
				      \draw[->](5,0.35)node[above]{$(k\!+\!1)_{th}$}--(5,0);
				      \fill (3,0)node[below]{$W_k$}circle(1pt);
				      \fill (4,0)node[below]{$t$}circle(1pt);
				      \draw (6.5,0)node[above right]{$N(t)=k$但$W_k<t$};
				      \draw (6.5,0)node[below right]{$W_k<t$但$N(t)=k$};
			      \end{tikzpicture}
		      \end{figure}
		      故$\{N(t)\leqslant k\}\nRightarrow \{W_k\leq t\}$
		\item $\displaystyle \{N(t) > k \} = \{N(t) \geqslant k+1\} = \{W_{k+1} \leqslant t\} \neq \{W_k < t\}$
		      \\ 同样参照上图,$\{W_k<t\}\nRightarrow \{N(t)>k\}$
	\end{enumerate}
\end{solution}
