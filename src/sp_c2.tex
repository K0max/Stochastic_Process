%第 5 题未解答


2.1 $N(t)$为一Possion过程, 对$s < t$试求条件概率$P\{N(s) = k | N(t) = n\}$.\\
	解:
	\begin{align*}
	P\{N(s)=k|N(t)=n\} & = P(N(s)=k, N(t)=n) / P(N(t)=n)\\
						& = P(N(s)-N(0)=k, N(t)-N(s)=n-k) / P(N(t)=n)\\ 
						& = \left[\frac{({\lambda s})^ke^{-{\lambda s}}}{k!} \cdot \frac{\left(\lambda(t-s)\right)^{n-k} e^{-\lambda(t-s)}}{(n-k)!}\right] \Bigg/ \left[\frac{(\lambda t)^n e^{-\lambda t}}{n!}\right]\\
						& = \frac{n!}{k!(n-k)!}\left(\frac{s}{t}\right)^k \left(1-\frac{s}{t}\right)^{n-k}\\
	\end{align*}

2.2 $\{N(t), t \geqslant 0\}$为一强度是$\lambda$的Possion过程. 对$s > 0$试计算$E[N(t) \cdot N(t+s)]$. \\
	解:
	\[
	\begin{aligned}
	\text{原式} & = E\left[N(t)\left(N(t+s)-N(t)+N(t)\right)\right]\\
		& = E\left(N(t)\right)E\left(N(t+s)-N(t)\right)+E\left(N^2(t)\right)\\
		& = \lambda t \cdot \lambda s + VarN(t) + E^2N(t)\\
		& = {\lambda}^2ts+\lambda t + (\lambda t)^2\\
		& = {\lambda}^2t(s+t)+\lambda t
	\end{aligned}
	\]

2.3 电报依平均速率为每小时3个的Possion过程到达电报局, 试问:\\
	(i)从早上八时到中午没收到电报的概率;\\
	(ii)下午第一份电报到达时间的分布是什么?\\
	解:\\
	(i)令$t$的计时单位为小时, 并以早上8:00为起始时刻\\
		所求事件即\{N(4) = 0\}\\
		其概率为$\frac{e^{-12} \cdot 12^0}{0!} \approx 6.1 \times 10^{-6}$\\
	(ii)取中午12:00为起始时刻, $T$表示下午第一份电报到达时间
		$F(t) = P(T \leqslant t) = P\left(N(t) \geqslant 1\right) = 1 - e^{-3t}$\\
		$\therefore f(t) = 3e^{-3t}$, 即$T$服从参数为3的指数分布\\

2.4 $\{N(t), t \geqslant 0\}$为一$\lambda = 2$的Possion过程, 试求:\\
	(i)$P\{N(1) \leqslant 2\}$;\\
	(ii)$P\{N(1) = 1 \text{且}N(2) = 3\}$;\\
	(iii)$P\{N(1) \geqslant 2 | N(1) \geqslant 1\}$.\\
	解:\\
	(i)\begin{align*}
	\text{原式} & = P\{N(1) - N(0) = 0\} + P\{N(1) - N(0) = 1\} + P\{N(1) - N(0) = 2\}\\
						& = e^{-2} + \frac{2e^{-2}}{1!}+\frac{2^2e^{-2}}{2!}\\
						& = 5e^{-2}
	\end{align*}
	(ii)\begin{align*}
	\text{原式} & = P\{N(1) - N(0) = 1 \text{且}N(2) - N(1) = 2\}\\
				& = P\{N(1) - N(0) = 1\} \cdot P\{N(2) - N(1) = 2\}\\
				& = 2e^{-2} \cdot 2e^{-2}\\
				& = 4e^{-4}
	\end{align*}
	(iii)\begin{align*}
	\text{原式} & = \frac{P\{N(1) \geqslant 2\} \text{且} N(1) \geqslant 1}{P\{N(1) \geqslant 1\}}\\
				& = \frac{P\{N(1) \geqslant 2\}}{P\{N(1) \geqslant 1\}}\\
				& = \frac{1-P\{N(1)-N(0) \leqslant 1\}}{1-P\{N(1)-N(0)=0\}}\\
				& = \frac{1-3e^{-2}}{1-e^{-2}}\\
	\end{align*}


% 2.5 证明概率$P_m(t) = P\{N(t) = m\}$在命题2.1的假定$(1)\sim(4)$下满足微分方程
% 	\[
% 	P^{\prime}_m(t) = -\lambda P_m(t) + \lambda P_{m-1}(t), m = 1,2,\cdots
% 	\]
% 	并证明在初始条件下$P_m(0) = 0, m = 1,2,\cdots$下的解为$\frac{{\lambda}^mt^m}{m!}e^{-\lambda t}$.\\


2.6 一部600页的著作总共有240个印刷错误, 试利用Possion过程近似求出某连续三页无错误的概率.\\
	解:
	设$N(t)$——到第$t$页为止的印刷错误数, 则$N(t)$为一Possion过程\\
	观点一:该Possion过程的参数$\lambda$未知\\
	则所求概率为
	\begin{align*}
	& \quad P\{N(t+3) - N(t) = 0 | N(600) - N(0) = 240\}\\
	& = P\{N(t+3)-N(t)=0;N(600)-N(0)=240\}/P\{N(600)-N(0)=240\}\\
	& = P\{N(3)-N(0)=0\}\cdot P\{N(600)-N(3)=240\}/P\{N(600)-N(0)=240\}\\
	& = e^{-3\lambda}\cdot \frac{(597\lambda)^{240}}{240!}e^{-597\lambda} \Bigg/ \bigg[ \frac{(600\lambda)^{240}}{240!} \cdot e^{-600\lambda} \bigg]\\
	& = \left(\frac{597}{600}\right)^{240} \approx 0.3003 \approx 0.3
	\end{align*}

	观点二:该Possion过程的参数$\lambda$已知, $\lambda = 240/600 = 0.4$, \\
	则所求概率为$P\{N(3) = 0\} = e^{-0.4 \times 3} \approx 0.3012 \approx 0.3$

2.7 $N(t)$是强度为$\lambda$的Possion过程. 给定$N(t) = n$, 试求第$r$个事件$(r \leqslant n)$发生的时刻$W_r$的条件概率密度$f_{W_r|N(t)=n}(w_r|n)$.\\
	解:
	\begin{align*}
	& \quad f_{W_r|N(t)=n}(w_r|n) \cdot \Delta w_r\\
	& = P\{N(w_r) - N(0) = r - 1, N(w_r+\Delta w_r)-N(w_r)=1|N(t) = n\}\\
	& = P\{N(w_r) - N(0) = r - 1\} \cdot P\{N(w_r+\Delta w_r)-N(w_r)=1\}\\
	& \qquad \cdot P\{N(t)-N(w_r+\Delta w_r)=n-r\} / P\{N(t) = n\}\\
	& = \frac{(\lambda w_r)^{r-1}}{(r-1)!}e^{-\lambda w_r} \cdot \left(\lambda \Delta w_r + o(\Delta w_r)\right) \cdot \frac{\left(\lambda (t-w_r-\Delta w_r)\right)^{n-r}}{(n-r)!} e^{-\lambda(t-w_r-\Delta w_r)} \Bigg/ \bigg[\frac{(\lambda t)^n}{n!} e^{\lambda t}\bigg]\\
	\end{align*}
	两边除以$\Delta w_r$并令$\Delta w_r \rightarrow 0$得\\
	$f_{W_r|N(t)=n}(w_r|n) = \frac{n!}{(r-1)!(n-r)!}\frac{(w_r)^{r-1}(t-w_r)^{n-r}}{t^n}$\\


2.8 令$\{N_i(t), t\geqslant 0\}, i = 1,2,\cdots, n$为n个独立的有相同强度参数$\lambda$的Possion过程. 记$T$为在全部$n$个过程中至少发生了一件事的时刻, 试求$T$的分布. \\
	解:
	\begin{align*}
	F_T(t) = P\{T \leqslant t\} & = 1 - P(T<t)\\
							& = 1 - P(N_i(t) = 0, i= 1, 2, \cdots, n)\\
							& = 1 - \big(e^{-\lambda t}\big)^n\\
	\end{align*}
	$\therefore f_T(t) = F^{\prime}_T(t) = n\lambda e^{-n\lambda t}$\\
	即$T$服从参数为$n\lambda$的指数分布.


% 2.9 考虑参数为$\lambda$的Possion过程$N(t)$, 若每一事件独立地以概率$p$被观察到, 并将观察到的过程记为$N_1(t)$. 试问$N_1(t)$是什么过程? $N(t)-N_1(t)$呢? $N_1(t)$与$N(t) - N_1(t)$是否独立?\\
% 解:



2.10 到达某加油站的公路上的卡车服从参数为${\lambda}_1$的Possion过程$N_1(t)$, 而到达的小汽车服从参数为${\lambda}_2$的Possion过程$N_2(t)$, 且过程$N_1(t)$与$N_2(t)$独立. 试问随机过程$N(t) = N_1(t) + N_2(t)$是什么过程? 并计算在总车流数$N(t)$中卡车首先到达的概率. \\
解:
\[
\begin{split}
g_{N(t)}(v) & = g_{N_1{t}}(v) \cdot g_{N_2{t}}(v)\\
		& = e^{\lambda_1 v(e^t-1)}e^{\lambda_2 v(e^t-1)}\\
		& = e^{(\lambda_1 +\lambda_2)v(e^t-1)}
\end{split}
\]
$\therefore N(t)$是强度为$\lambda_1+\lambda_2$的Possion过程\\
记$W_1,W_2$分别为卡车、小汽车的第一次到达时间, 则$W_1$服从参数为$\lambda_1$的指数分布, $W_1$服从参数为$\lambda_1$的指数分布.
\[
\begin{split}
\therefore P(W_1 < W_2) & = \iint_{0 \leqslant W_1 < W_2 \leqslant +\infty}f_{W_1,W_2}(w_1,w_2)\,dw_1dw_2\\
					& = \int^{+\infty}_0dw_1\int^{+\infty}_{w_1}\lambda_1\lambda_2e^{-w_1\lambda_1-w_2\lambda_2}\,dw_2\\
					& = \frac{\lambda_1}{\lambda_1+\lambda_2}
\end{split}
\]


2.11 冲击模型(Shock Model)   记$N(t)$为某系统到某时刻$t$受到的冲击次数, 它是参数为$\lambda$的Possion过程. 设第k次冲击对系统的损害大小$Y_k$服从参数为$\mu$的指数分布, $Y_k, k = 1, 2, \cdots, $独立同分布. 记$X(t)$为系统所受到的总损害. 当损害超过一定的极限$\alpha$时系统不能运行, 寿命终止, 记$T$为系统寿命. 试求该系统的平均寿命$ET$, 并对所得结果作出直观解释.\\
提示:对非负随机变量$\E T = \int^\infty_0P(T>t)\,dt$\\
解:令$W_n = \sum\limits^n_{k=1}Y_k$\\
法一:
\[
\begin{split}
P(T>t) & = P\{X(t) \leqslant \alpha\}\\
	& = P\{\sum^{N(t)}_{k=1}Y_k \leqslant \alpha\}\\
	& = \sum^{+\infty}_{n=0}P\Bigg\{\sum^{N(t)}_{k=1}Y_k \leqslant \alpha | N(t) = n\Bigg\} \cdot P\{N(t) = n\}\\
	& = \sum^{+\infty}_{n=0}P\Bigg\{\sum^n_{k=1}Y_k \leqslant \alpha | N(t) = n\Bigg\} \cdot P\{N(t) = n\}\\
	& = \sum^{+\infty}_{n=0}P\{W_n \leqslant \alpha\} \cdot P\{N(t) = n\}\\
\end{split}
\]
求和式中当$n=0$时认为$P\{W_n \leqslant \alpha | N(t) = n\} = 1$\\
$\because Y_k \sim \exp(\mu), ~\therefore W_n = \sum\limits^n_{k=1}Y_k \sim \Gamma(n,\mu)$\\
\[
\begin{split}
\therefore & ~ P(W_n \leqslant \alpha) = \frac{\mu^n}{(n-1)!}\int^\alpha_0s^{n-1}e^{-\mu s}\,ds~~~~(n\geqslant 1)\\
& ~ P\{N(t) = n\} = \frac{(\lambda t)^n}{n!}e^{-\lambda t}
\end{split}
\]
\[
\therefore P(T>t) = e^{-\lambda t} + e^{-\lambda t}\sum^{+\infty}_{n=1}\frac{(\lambda\mu t)^n}{n!(n-1)!}\int^\alpha_0s^{n-1}e^{-\mu s}\,ds
\]
\[
\begin{split}
\therefore \E T & = \int^{+\infty}_0P(T>t)\,dt\\
				& = \frac{1}{\lambda} + \sum^{+\infty}_{n=1}\frac{(\lambda\mu)^n}{n!(n-1)!}\int^{+\infty}_0t^ne^{-\lambda t}\,dt\int^\alpha_0s^{n-1}e^{-\mu s}\,ds\\
				& = \frac{1}{\lambda} + \frac{1}{\lambda}\sum^{+\infty}_{n=1}\frac{\mu^n\Gamma(n+1)}{n!(n-1)!}\int^\alpha_0s^{n-1}e^{-\mu s}\,ds\\
				& = \frac{1}{\lambda} + \frac{1}{\lambda}\int^\alpha_0\bigg[\sum^{+\infty}_{n=1}\frac{(\mu s)^{n-1}e^{-\mu s}}{(n-1)!}\bigg]\,d(\mu s)\\
				& = \frac{1+\mu\alpha}{\lambda}
\end{split}
\]
从结果看, 若$\lambda$越大(系统所受冲击越频繁), $\mu$越小(每次冲击所造成的平均损害越大), $\alpha$越小(系统所能承受的的损害极限越小), 则系统平均寿命越短, 且当$\alpha$等于$0$时系统的平均寿命即为第一次冲击到来的平均时间, 符合常识.\\
法二:
\[
\begin{split}
\because G_n(\alpha) & = P\{Y_1+\cdots+Y_k \leqslant \alpha\}\\
					& = P\{W_n \leqslant \alpha\}\\
					& = P\{N_1(\alpha) \geqslant n\}\\
					& = \sum^{+\infty}_{k=n}\frac{(\mu\alpha)^n}{k!}e^{-\mu\alpha}
\end{split}
\]
其中$N_1(t)$是强度为$\mu$的Possion过程\\
\[
\begin{split}
\therefore P(T>t) & = P\{X(t) \leqslant \alpha\}\\
			& = \sum^{+\infty}_{n=0}\frac{(\lambda t)^ne^{-\lambda t}}{n!}G_n(\alpha)~~~~(\text{例}2.4)\\
			& = \sum^{+\infty}_{n=0}\sum^{+\infty}_{k=n}\frac{(\lambda t)^n}{n!}e^{-\lambda t}\frac{(\mu\alpha)^n}{k!}e^{-\mu\alpha}\\
			& = \sum^{+\infty}_{k=0}\frac{(\mu\alpha)^n}{k!}e^{-\mu\alpha}\sum^{k}_{n=0}\frac{(\lambda t)^n}{n!}e^{-\lambda t}
\end{split}
\]
\[
\begin{split}
\therefore \E T & =  \int^{+\infty}_0P(T>t)\,dt\\
				& = \sum^{+\infty}_{k=0}\frac{(\mu\alpha)^k}{k!}e^{-\mu\alpha}\sum^{k}_{n=0}\int^{+\infty}_0\frac{(\lambda t)^n}{n!}e^{-\lambda t}\,dt\\
				& = \sum^{+\infty}_{k=0}\frac{(\mu\alpha)^k}{k!}e^{-\mu\alpha}\frac{(k+1)\Gamma(n+1)}{n!\lambda}\\
				& = \frac{1}{\lambda}\sum^{+\infty}_{k=0}\frac{(\mu\alpha)^k}{k!}e^{-\mu\alpha} + \frac{\mu\alpha}{\lambda}\sum^{+\infty}_{k=1}\frac{(\mu\alpha)^{k-1}}{(k-1)!}e^{-\mu\alpha}\\
				& = \frac{1+\mu\alpha}{\lambda}
\end{split}
\]



2.12 令$N(t)$是强度函数为$\lambda (t)$的非齐次Possion过程, $X_1, X_2, \cdots$为事件间的时间间隔.
(i)$X_i$是否独立;\\
(ii)$X_i$是否同分布;\\
(iii)试求$X_1$与$X_2$的分布.\\
解:
记$m(t) = \int^t_0\lambda(u)\,du$\\
法一:
等待时间$W_1,W_2$的联合分布函数为
\[
\begin{split}
F_{W_1,W_2}(t_1,t_2) & = P(W_1 \leqslant t_1, W_2 \leqslant t_2),~~~~0 \leqslant t_1 < t_2\\
					& = P(N(t_1) \geqslant 1, N(t_2) \geqslant 2)\\
					& = \sum^{+\infty}_{k=2}\sum^k_{\ell=1}P(N(t_1) = \ell, N(t_2) = k)\\
					& = \sum^{+\infty}_{k=2}\sum^k_{\ell=1}P(N(t_1) = \ell, N(t_2) - N(t_1) = k - \ell)\\
					& = \sum^{+\infty}_{k=2}\sum^k_{\ell=1}P(N(t_1) = \ell)P(N(t_2) - N(t_1) = k - \ell)\\
					& = \sum^{+\infty}_{k=2}\sum^k_{\ell=1}\frac{(m(t_1)^\ell)}{\ell !}e^{-m(t_1)}\cdot \frac{[m(t_2)-m(t_1)]^{k-\ell}}{(k-\ell)!}e^{-[m(t_2)-m(t_1)]}\\
					& = \sum^{+\infty}_{k=2}\sum^k_{\ell=1}\frac{(m(t_1))^\ell[m(t_2)-m(t_1)]^{k-\ell}}{\ell !(k-\ell)!}e^{-m(t_2)}\\
					& = e^{-m(t_2)}\sum^{+\infty}_{k=2}\frac{1}{k!}\sum^k_{\ell=1}C^\ell_k(m(t_1))^\ell[m(t_2)-m(t_1)]^{k-\ell}\\
					& = e^{-m(t_2)}\sum^{+\infty}_{k=2}\frac{1}{k!}\bigg\{\sum^k_{\ell=0}C^\ell_k(m(t_1))^\ell[m(t_2)-m(t_1)]^{k-\ell}-[m(t_2)-m(t_1)]^k\bigg\}\\
					& = e^{-m(t_2)}\Big\{e^{m(t_2)}-e^{m(t_2)-m(t_1)}-m(t_1)\Big\}\\
					& = 1-e^{-m(t_1)}-m(t_1)e^{-m(t_2)}\\
\end{split}
\]
\[
\therefore f_{W_1,W_2}(t_1,t_2) = \frac{\partial^2F_{W_1,W_2}(t_1,t_2)}{\partial t_1\partial t_2} = \lambda(t_1)\lambda(t_2)e^{-m(t_2)}
\]
\[
\because
\begin{cases}
W_1 = X_1\\
W_2 = X_1 + X_2
\end{cases}
\]
\[
\therefore f_{X_2,X_2}(t_1,t_2) = \lambda(t_1)\lambda(t_1+t_2)e^{-m(t_1+t_2)}\text{不能写为}g_1(t_1)g_2(t_2)\text{形式}
\]
$\therefore X_1, X_2$不独立\\
\[
\begin{split}
\text{又有} f_{X_1}(t_1) & = \lambda(t_1)\int^{+\infty}_0\lambda(t_1+t_2)e^{-m(t_1+t_2)}\,dt_2\\
						& = \lambda(t_1)\Big[e^{-m(t_1)}-e^{-m(+\infty)}\Big],~~~~~t_1> 0\\
\end{split}
\]
下面确定$e^{-m(\infty)}$:\\
\[
1 = \int^{+\infty}_0f_{X_1}(t_1)\,dt_1 = \int^{+\infty}_0\lambda(t_1)\Big[e^{-m(t_1)}-e^{-m(+\infty)}\Big]\,dt_1 = 1 - [m(+\infty)+1]e^{-m(+\infty)}
\]
$\therefore e^{-m(+\infty)} = 0$\\
\[
\begin{split}
\therefore f_{X_1}(t_1) & = \lambda(t_1)e^{-m(t_1)}~~~(t_1>0)\\
f_{X_2}(t_2) & = \int^{+\infty}_0\lambda(t_1)\lambda(t_1+t_2)e^{-m(t_1+t_2)}\,dt_1~~~(t_2>0)\\
\end{split}
\]
$\therefore X_1,X_2$不同分布且其概率密度函数如上.\\
法二:
\[
\begin{split}
P\{X_2 > t | X_1 = s\} & = \lim_{s'\rightarrow s}P\big\{N(s+t) - N(s) = 0 | N(s') = 0, N(s) - N(s') = 1\big\}\\
					& = P\big\{N(s+t) - N(s) = 0 \big\}\\
					& = e^{m(s)-m(s+t)}
\end{split}
\]
与$s$有关, 故$X_1, X_2$不独立.\\
\[
P\{X_1 > t\} = P\{N(t) = 0\} = e^{-m(t)}
\]
\[
\begin{split}
P\{X_2 > t\} & = \int^{+\infty}_0P\{X_2 > t | X_1 = s\}f_{X_1}(s)\,ds\\
			& = \int^{+\infty}_0e^{-m(s+t)}\lambda(s)\,ds\\
			& = e^{-m(t)}\int^{+\infty}_0e^{m(t)-m(s+t)}\lambda(s)\,ds\\
			& = P\{X_1 > t\}\int^{+\infty}_0e^{m(t)-m(s+t)}\lambda(s)\,ds\\
			& \not\equiv P\{X_1 > t\}
\end{split}
\]
$\therefore X_1,X_2$不同分布($P\{X_1 > t\text{和}P\{X_2 > t\}$给出$X_1\text{和}X_2$的分布).


2.13 考虑对所有$t$, 强度函数$\lambda (t)$均大于$0$的非齐次Possion过程$\{N(t), t \geqslant 0\}$. 令$m(t) = \int^t_0 \lambda(u)\, du, m(t)$的反函数为$\ell(t)$, 记$N_1(t)\equiv N(\ell(t))$. 试证$N_1(t)$是通常的Possion过程, 试求$N_1(t)$的强度参数$\lambda$.\\
解:\\
(i)$N_1(0)=N(\ell(0))=N(0)=0$\\
(ii)$\because m(\ell)$单增, $\therefore \ell(t)$单增\\
	$\therefore $对任意$0 \leqslant t_1 < t_2 \cdots < t_n$, 有$0 \leqslant \ell(t_1) < \ell(t_2) < \cdots < \ell(t_n)$\\
	\begin{align*}
	\therefore \text{有} N_1(t_2) - N_1(t_1) & = N(\ell(t_2)) - N(\ell(t_1))\\
				N_1(t_3) - N_1(t_2) & = N(\ell(t_3)) - N(\ell(t_2))\\
				\vdots \\
				N_1(t_n) - N_1(t_{n-1}) & = N(\ell(t_n)) - N(\ell(t_{n-1}))\\
	\end{align*}
	$\because N(t)$是独立增量过程\\
	$\therefore N_1(t)$也是独立增量过程\\
(iii)$\forall 0 \leqslant s < t$, 有\\
	\[
	\begin{split}
	P(N_1(t) - N_1(s) = k) & = P\{N(\ell(t)) - N(\ell(s)) = k\}\\
						& = \frac{[m(\ell(t)) - m(\ell(s))]^k}{k!} e^{-[m(\ell(t)) - m(\ell(s))]}\\
						& = \frac{(t-s)^k}{k!}e^{-(t-s)} ~~~~~~~k=0,1,\cdots
	\end{split}
	\]
	$\therefore N_1(t)$是强度为$1$的Possion过程


2.14 设$N(t)$为更新过程, 试判断下述命题的真伪:\\
(i)$\{N(t) < k \} \Longleftrightarrow \{W_k > t\}$;\\
(ii)$\{N(t) \leqslant k \} \Longleftrightarrow \{W_k \geqslant t\}$;\\
(iii)$\{N(t) > k \} \Longleftrightarrow \{W_k < t\}$;\\
其中$W-k$为第$k$个事件的等待时间.\\
解:\\
(i)\[
	\begin{split}
	\{N(t) < k \} & = \overline{\{N(t) \geqslant k\}}\\
	 			& = \overline{\{W_k \leqslant t\}}\\
				& = \{W_k > t\}\\
	\end{split}
	\]
(ii)\[
	\begin{split}
	\{N(t) \leqslant k \} & = \{N(t) < k + 1\}\\
				& = \overline{\{N(t) \geqslant k + 1\}}\\
				& = \overline{\{W_{k+1} \leqslant t\}}\\
				& = \{W_{k+1} > t\}\\
				& \neq \{W_k \geqslant t\}\\
	\end{split}
	\]
(iii)\[
	\begin{split}
	\{N(t) > k \} & = \{N(t) \geqslant k+1\}\\
				& = \{W_{k+1} \leqslant t\}\\
				& \neq \{W_k < t\}\\
	\end{split}
	\]


2.(14.5) 设$X_1(t)$与$X_2(t)$为两个独立的Possion过程, 速率分别为$\lambda_1$和$\lambda_2$\\
(a)试证$X_1(t) + X_2(t)$为速率$\lambda_1+\lambda_2$的Possion过程\\
(b)~$X_1(t) - X_2(t)$是Possion过程吗?为什么?\\
(c)~试求在过程$X_1(t)$的任意两个相邻事件时间间隔之内, 过程$X_2(t)$恰好发生$k$个事件的概率$P_k(k\geqslant 0)$\\
解:\\
(a)
\[
\begin{split}
g_{X_1(t)+X_2(t)}(v) & = g_{X_1(t)}(v)\cdot g_{X_2(t)}(v)\\
					& = \exp\Big\{\lambda_1 t(e^v-1)\Big\}\cdot \exp\Big\{\lambda_2 t(e^v-1)\Big\}\\
					& = \exp\Big\{(\lambda_1+\lambda_2)t(e^v-1)\Big\}\\
\end{split}
\]
$\therefore X_1(t) + X_2(t)$是强度为$\lambda_1+\lambda_2$的Possion过程.\\
(b)
\[
\begin{split}
g_{X_1(t)-X_2(t)}(v) & = g_{X_1(t)}(v)\cdot g_{X_2(t)}(v)\\
					& = \exp\Big\{\lambda_1 t(e^v-1)\Big\}\E\Big(e^{-vX_2(t)}\Big)\\
					& = \exp\Big\{\lambda_1 t(e^v-1)\Big\}\cdot \exp\Big\{\lambda_2 t(e^{-v}-1)\Big\}\\
\end{split}
\]
不能化为Possion过程的矩母函数形式,\\
$\therefore X_1(t) - X_2(t)$不是Possion过程.\\
(c)$X_1(t)$的相邻两个事件时间间隔$\tau$服从参数为$\lambda_1$的指数分布, 其概率密度函数为$f(\tau) = \lambda_1 e^{-\lambda_1\tau}$\\
\[
\begin{split}
\therefore P_k & = \int^{+\infty}_0	f(\tau)P\Big\{X_2(t+\tau)-X_2(t) = k\Big\}\,d\tau\\
			& = \int^{+\infty}_0 \lambda_1 e^{-\lambda_1\tau}\cdot \frac{(\lambda_2\tau)^k}{k!}e^{-\lambda_2\tau}\,d\tau\\
			& = \frac{\lambda_1\lambda^k_2}{(\lambda_1+\lambda_2)^{k+1}k!}\cdot \int^{+\infty}_0 \big((\lambda_1+\lambda_2)\tau\big)^ke^{-(\lambda_1+\lambda_2)\tau}\,d(\lambda_1+\lambda_2)\tau\\
			& = \frac{\lambda_1}{\lambda_1+\lambda_2}\bigg(\frac{\lambda_2}{\lambda_1+\lambda_2}\bigg)^k
\end{split}
\]
