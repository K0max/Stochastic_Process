\section{第四章~平稳过程}
\noindent\textcolor{red}{以下如果没有指明变量$t$的取值范围,一般视为$t\in \mathbb{R}$,平稳过程是指宽平稳过程.}

\begin{problem}{4.1}
设$X(t)=\sin Ut$,这里$U$为$(0,2\uppi)$上的均匀分布.
\begin{enumerate}[label=(\alph*)]
	\item 若$t=1,2,\cdots$,证明$\{X(t),t=1,2,\cdots \}$是宽平稳但不是严平稳过程,
	\item 设$t\in [0,+\infty )$,证明$\{X(t),t\geqslant 0\}$既不是严平稳也不是宽平稳过程.
\end{enumerate}
\end{problem}
\begin{solution}
	\begin{enumerate}[label=(\alph*)]
		\item \[\E[X(t)] = \E(\sin Ut) = \int^{2\uppi}_0 \frac{1}{2\uppi}\sin Ut \d U = 0 \qquad (t = 1,2,\cdots)\]
		      \[
			      \begin{split}
				      \cov(X(t), X(s)) & = \E(\sin Ut \cdot \sin Us)\\
				      & = \frac{1}{2}\E\left[\cos(t-s)U-\cos(t+s)U\right]\\
				      & = \frac{1}{4\uppi}\left\{\frac{1}{t-s}\sin(t-s)U \Big|^{2\uppi}_0 - \frac{1}{t+s}\sin(t+s)U \Big|^{2\uppi}_0\right\}\\
				      & = 0 \qquad (t\neq s)
			      \end{split}
		      \]
		      当$t=s$时$\cov(X(t), X(s)) = \E(\sin^2 Ut) = \frac{1}{2}$ \quad $\therefore$ 是宽平稳\\
		      考虑$F_t(x) = \p(\sin Ut \leqslant x)$, 显然$F_{t+h} = \p[\sin U(t+h) \leqslant x]$与其不一定相同 \quad $\therefore$ 不是严平稳
		\item \[\E[X(t)] = \frac{1}{2\uppi t}(1-\cos 2\uppi t)\]
		      \[\var[X(t)] = \E\left(\sin Ut - \frac{1}{2\uppi t}(1-\cos 2\uppi t)\right)^2 = \frac{1}{2} - \frac{\sin 4\uppi t}{8\uppi t} - \left(\frac{1-\cos 2\uppi t}{2\uppi t}\right)^2\]
		      都与$t$相关 \quad $\therefore$ 不是宽平稳\\
		      若其严平稳, 则因二阶矩存在, 应为宽平稳, 矛盾.\quad $\therefore$ 不是严平稳.\\
	\end{enumerate}
\end{solution}

\begin{problem}{4.2}
设$\{X_n, n = 0,1,2\cdots\}$ 是平稳序列, 定义$\{X^{(i)}_n, n = 1,2,\cdots\}, i = 1, 2, \cdots, $为
\[\begin{aligned}
		X^{(1)}_n & = X_n - X_{n-1},             \\
		X^{(2)}_n & = X^{(1)}_n - X^{(1)}_{n-1}, \\
		          & \cdots\cdots
	\end{aligned}
\]
证明这些序列仍是平稳序列.\\
\end{problem}
\begin{solution}
	\begin{enumerate}[label=$\arabic*^\circ$]
		\item $\ell = 0$时,$\E(X_n)$依定义为常数$C_0$\\
		      $\cov(X_n, X_m)$依定义为$n-m$的函数$f_0(n-m)\Rightarrow$ 成立
		\item 设当$\ell \leqslant k$时成立, 则当$\ell = k + 1$时
		      \[\E X^{(\ell)}_n = \E(X^{(k)}_n - X^{(k)}_{n-1}) = C_k - C_k = 0\]
		      \[
			      \begin{split}
				      \cov(X^{(k+1)}, X^{(k+1)}_m) & = \E(X^{(k+1)}_n X^{(k+1)}_m)\\
				      & = \E\left[(X^{(k)}_n - X^{(k)}_{n-1})(X^{(k)}_m - X^{(k)}_{m-1})\right]\\
				      & = \E(X^{(k)}_n X^{(k)}_m) - \E(X^{(k)}_{n-1}X^{(k)}_m) - \E(X^{(k)}_{n}X^{(k)}_{m-1}) + \E(X^{(k)}_{n-1}X^{(k)}_{m-1})\\
				      & = f_k(n-m) - f_k(n-1-m) - f_k(n-m+1) + f_k(n-m)\\
				      & = f_{\ell}(n-m)
			      \end{split}
		      \]
		      只与$n-m$有关\quad $\therefore$是平稳的
	\end{enumerate}
\end{solution}

\begin{problem}{4.3}
设$X_n = \sum\limits^N_{k=1}\sigma_k\sqrt{2}\cos(a_kn-U_k)$, 这里$\sigma_k$和$a_k$为正常数, $k=1, \cdots, N;U_1, \cdots, U_n$是$(0,2\uppi)$上独立均匀分布随机变量, 证明$\{X_n, n = 0, \pm 1, \cdots\}$是平稳过程.
\end{problem}
\begin{solution}
	\[
		\begin{split}
			\E(X_n) & = \E\left[\sum^N_{k=1}\sigma_k\sqrt{2}(\cos(a_k n)\cos U_k +\sin(a_k n)\sin U_k)\right]\\
			& = \sum_{k=1}^{N}\sigma_k \sqrt{2}\big[\E(\cos U_k)\cos a_k n + \E(\sin U_k)\sin a_k n\big]\\
			& = 0\\
			\cov(X_n, X_m) & = \E(X_n X_m)-\E(X_n)\E(X_m) = \E(X_n X_m)\\
			& = \E\left[\sum^N_{k=1}\sigma_k\sqrt{2}\cos(a_k n-U_k)\sum^N_{j=1}\sigma_j\sqrt{2}\cos (a_j m-U_j)\right]\\
			& = \sum^N_{k=1}2\sigma^2_k\E\left[\cos(a_kn-U_k)\cos(a_km-U_k)\right] + \sum_{k\neq j}2\sigma_k\sigma_j\E[\cos(a_k n-U_k)]\E[\cos(a_j m-U_j)]\\
			& = \sum^N_{k=1}\sigma^2_k\E\left[\cos(a_k(n-m)) + \cos(a_kn + a_km - 2U_k)\right] + 0\\
			& = \sum^N_{k=1}\sigma^2_k\cos[a_k(n-m)]\\
		\end{split}
	\]
	只与$n-m$有关\quad $\therefore$宽平稳.
\end{solution}

\begin{problem}{4.4}
设$A_k, k = 1,2,\cdots,n$是$n$个实随机变量; $\omega_k, k = 1,2,\cdots,n,$是$n$个实数. 试问$A_k$以及$A_k$之间应满足怎样的条件才能使
\[Z(t) = \sum^n_{k=1}A_k \e^{j\omega_kt}\]
是一个复的平稳过程.
\end{problem}
\begin{solution}
	要求
	\[\E[Z(t)] = \sum^n_{k=1}\E(A_k \e^{j\omega_kt}) = const\]
	$\therefore \E(A_k) = 0$,要求
	\[\cov(Z(t), Z(s)) = \E[Z(t)\overline{Z(s)}] = \sum^{\infty}_{k=1}\sum^{\infty}_{\ell=1}\E(A_k A_\ell)\cdot \e^{j\omega_kt - j\omega_\ell s}\]
	只与$t-s$有关\\
	$\therefore \E(A_k A_l) = 0 \quad (k\neq \ell \text{\,且\,}\omega_k \neq \omega_\ell)$
\end{solution}

\begin{problem}{4.5}
设$\{X_n, n=1,2,\cdots \}$是一列独立同分布的随机变量序列,$\p(X_n=1)=p,\p(X_n=-1)=1-p,\quad n=1,2,\cdots $,令$S_0=0, S_n=(X_1+\cdots +X_n)/\sqrt{n},\quad n=1,2,\cdots $,求随机序列$\{S_n=1,2,\cdots \}$的协方差函数和自相关函数.$p$取何值时此序列为平稳序列?
\end{problem}
\begin{solution}
	由题意$\E(X_n)=2p-1,\E(X_n^2)=1,\E(S_n)=\frac{1}{\sqrt{n}}\sum_{i=1}^{n}\E(X_1) = \sqrt{n}(2p-1)$
	\begin{enumerate}[label=(\roman*)]
		\item \[\begin{split}
				      \cov(S_n,S_m) &= \E(S_n S_m) = \E(S_n)\E(S_m)\\
				      &= \frac{1}{\sqrt{mn}}\sum_{i=1}^{n}\sum_{j=1}^{m}\E(X_i X_j) - \sqrt{mn}(2p-1)^2 \xlongequal{\text{不妨设}m<n} \\
				      &= \frac{1}{\sqrt{mn}}\left(\sum_{i=1}^{m}\E(X_i^2)+\sum_{(i,j)=(1,1),i\neq j}^{(i,j)=(m,n)}\E(X_i)\E(X_j)\right) - \sqrt{mn}(2p-1)^2\\
				      &= \frac{4m}{\sqrt{mn}}p(1-p)
			      \end{split}\]
		\item \[r_X(n,m)=\E(S_n S_m)=\frac{4\min {m,n}}{\sqrt{mn}}p(1-p)+\sqrt{mn}(2p-1)^2\]
		\item 若$\{S_n\}$平稳,则$\E(S_n)\equiv const$, 由$\sqrt{2}S_n = \sqrt{n}(2p-1) \Rightarrow p=\frac{1}{2}$
		      但此时$\cov(S_n,S_m) = \frac{\min \{m,n\}}{\sqrt{mn}}$与$m,n$有关,故不存在$p$使得序列平稳
	\end{enumerate}
\end{solution}

\begin{problem}{4.6}
设$\{X(t)\}$是一个平稳过程, 对每个$t \in \mathbb{R}$, $X'(t)$存在. 证明对每个给定的$t$, $X(t)$与$X'(t)$不相关, 其中$X'(t) = \frac{\d X(t)}{\d t}$.
\end{problem}
\begin{solution}
	以下假定求导数和求期望可交换\\
	设$\E[X(t)] = m, \var[X(t)] = \sigma^2$\\
	$\therefore \E[X(t+\Delta t)] = m$\\
	$\because X'(t) = \lim\limits_{\Delta t \to 0}\frac{X(t+\Delta t) - X(t)}{\Delta t}$\\
	$\therefore \E[X'(t)] = 0$
	\[
		\therefore \cov(X(t), X'(t)) = \E[X(t)X'(t)] = \frac{1}{2}\E\Big[\big(X^2(t)\big)'\Big] = \frac{1}{2}\big(\E[X^2(t)]\big)' = \frac{1}{2}(\sigma^2 + m^2)' = 0
	\]
	$\therefore$ 不相关
\end{solution}

\begin{problem}{4.7}
设$\{X(t)\}$是高斯过程, 均值为$0$, 协方差函数$R(\tau) = 4\e^{-2|\tau|}$. 令
\[Z(t) = X(t+1),\quad W(t) = X(t-1),\]
\begin{enumerate}[label=(\roman*)]
	\item 求$\E(Z(t)W(t))$和$\E(Z(t) + W(t))^2$;
	\item 求$Z(t)$的密度函数$f_Z(z)$及$\p(Z(t)<1)$;
	\item 求$Z(t), W(t)$的联合密度$f_{Z,W}(z,w)$.
\end{enumerate}
\end{problem}
\begin{solution}
	\begin{enumerate}[label=(\roman*)]
		\item \[
			      \begin{split}
				      \E[Z(t)W(t)] &= \E[X(t+1)X(t-1)] = R(2) = 4\e^{-4}\\
				      \E[Z(t)W(t)]^2 & = \E\big[X^2(t+1)+2X(t+1)X(t-1)+X^2(t-1)\big]\\
				      & = 2\E[X^2(t)]+2R(2)\\
				      & = 2\big\{\var[X(t)]-\E^2[X(t)]\big\} + 4\e^{-4}\\
				      & = 2R(0) + 4\e^{-4}\\
				      & = 4(1+\e^{-4})
			      \end{split}
		      \]
		\item $Z(t) = X(t+1) \sim N(0, 2^2)$
		      \[\begin{aligned}
				      \therefore f_Z(z)     & = \frac{1}{\sqrt{2\uppi \cdot 2^2}}\e^{-\frac{z^2}{2\cdot 2^2}} = \frac{1}{\sqrt{8\uppi}}\e^{-\frac{z^2}{8}} \\
				      \therefore \p[Z(t)<1] & = \int^1_{-\infty}f_Z(z)\d z = \frac{1}{\sqrt{8\uppi}}\int^1_{-\infty}\e^{-\frac{z^2}{8}}\d z
			      \end{aligned}\]
		\item 显然$f_{Z,W}(z,w)$为二维正态分布概率密度函数,协方差矩阵为
		      \[
			      \bm{C} =
			      \begin{pmatrix}
				      4        & 4\e^{-4} \\
				      4\e^{-4} & 4
			      \end{pmatrix}
		      \]
		      其逆矩阵
		      \[
			      \bm{C}^{-1}=
			      \begin{pmatrix}
				      \frac{1}{4(1-\e^{-8})}        & -\frac{\e^{-4}}{4(1-\e^{-8})} \\
				      -\frac{\e^{-4}}{4(1-\e^{-8})} & \frac{1}{4(1-\e^{-8})}
			      \end{pmatrix}
		      \]
		      其行列式$\left|\bm{C}\right| = 16(1-\e^{-8})$,期望向量$\bm{\bar{\mu }} = (0,0)$
		      \[
			      \begin{split}
				      \therefore f_{Z,W}(z,w) & = \frac{1}{2\uppi\left|\bm{C}\right|}\exp\Bigg\{-\frac{1}{2}\Big((z,w)-\bm{\bar{\mu}}\Big)\bm{C}^{-1}\Big((z,w)-\bm{\bar{\mu}}\Big)^T\Bigg\}\\
				      & = \frac{1}{8\uppi \sqrt{1-\e^{-8}}}\exp\Bigg\{-\frac{z^2+w^2-2\e^{-4}wz}{8(1-\e^{-8})}\Bigg\}
			      \end{split}
		      \]
	\end{enumerate}
\end{solution}

\begin{problem}{4.8}
设$\{X(t), t\in \mathbf R\}$是一个严平稳过程, $\varepsilon$为只取有限个值的随机变量. 证明$\{Y(t) = X(t-\varepsilon), t\in \mathbf R\}$仍是一个严平稳过程.\\
提示:对$\varepsilon$用全概率公式.
\end{problem}
\begin{solution}
	设$\varepsilon$可取$\varepsilon_1, \varepsilon_2, \cdots, \varepsilon_n$,则\\
	\[
		\begin{split}
			& \p\big\{Y(t_1+h) \leqslant y_1, \cdots, Y(t_k+h) \leqslant y_k\big\}\\
			=& \p\big\{X(t_1-\varepsilon+h) \leqslant y_1, \cdots, X(t_k-\varepsilon+h) \leqslant y_k\big\}\\
			=& \sum^n_{i=1}\p(\varepsilon = \varepsilon_i)\p\big\{X(t_1-\varepsilon_i + h) \leqslant y_1, \cdots, X(t_k-\varepsilon_i + h) \leqslant y_k | \varepsilon = \varepsilon_i\big\}
		\end{split}
	\]
	$\because X(t)$严平稳
	\[
		\begin{split}
			\therefore \text{上式} & = \sum^n_{i=1}\p(\varepsilon = \varepsilon_i)P\big\{X(t_1-\varepsilon_i) \leqslant y_1, \cdots, X(t_k-\varepsilon) \leqslant y_k | \varepsilon = \varepsilon_i\big\}\\
			& = \p\big\{X(t_1-\varepsilon) \leqslant y_1, \cdots, X(t_k-\varepsilon) \leqslant y_k\big\}\\
			& = \p\big\{Y(t_1) \leqslant y_1, \cdots, Y(t_k) \leqslant y_k\big\}\\
		\end{split}
	\]
	$\therefore Y(t)$为严平稳.
\end{solution}

\begin{problem}{4.9}
设$\{X(t),t\in \mathbb{R}\}$是一个严平稳过程,构造随机过程$Y$如下:$Y(t)=1$,若$X(t)>0;-1,$若$X(t)\leq 0$.证明$\{Y(t),t\in \mathbb{R}\}$是一个平稳过程.如果进一步假定$\{X(t),t\in \mathbb{R}\}$是均值为零的Gauss平稳过程,证明$R_Y(\tau)$为$\frac{2}{\uppi}\arcsin (R_X(\tau)/R_X(0))$.
\end{problem}
\begin{solution}
	本题暂缺.
\end{solution}

\begin{problem}{4.10}
设$\{X(t)\}$是一个复值平稳过程, 证明
\[E|X(t+\tau)-X(t)|^2 = 2\Re e(R(0)-R(\tau)).\]
\end{problem}
\begin{solution}
	记$m=\E[X(t)]$,则
	\[
		\begin{split}
			\E\big|X(t+\tau) - X(t)\big|^2 & = \E\big|(X(t+\tau)-m)-(X(t)-m)\big|^2\\
			& = \E\big|X(t+\tau)-m\big|^2 + \E\big|X(t)-m\big|^2 - \E\Big[(X(t+\tau)-m)\overline{(X(t)-m)}\,\Big] \\
			& \quad - \E\Big[(X(t)-m)\overline{(X(t+\tau)-m)}\,\Big]\\
			& = 2R(0)-R(-\tau)-R(\tau)
		\end{split}
	\]
	又$\because R(-\tau) = \overline{R(\tau)}\quad \therefore \text{上式} = 2\Re e(R(0)-R(\tau))$
\end{solution}

\begin{problem}{4.11}
设$\{X(t)\}$是零均值的平稳高斯过程, 协方差函数为$R(\tau)$, 证明
\[\p(X'(t) \leqslant a) = \phi\Bigg(\frac{a}{\sqrt{-R''(0)}}\Bigg),\]
其中$\phi(\cdot)$为标准正态分布函数.
\end{problem}
\begin{solution}
	注意到$X'(t)$服从正态分布\\
	而$\E[X'(t)] = \big\{\E[X(t)]\big\}' = 0$\\
	$\var(X'(t)) = \cov(X'(t), X'(t+0)) = -R''(0)$\\
	$\therefore X'(t) \sim N(0,-R''(0))$\\
	\[\therefore \p\big(X'(t) \leqslant a\big) = \p\Bigg(\frac{X'(t)}{\sqrt{-R''(0)}} \leqslant \frac{a}{\sqrt{-R''(0)}}\Bigg) = \phi\Bigg(\frac{a}{\sqrt{-R''(0)}}\Bigg)\]
\end{solution}

\begin{problem}{4.12}
设$\{X(t)\}$为连续宽平稳过程, 均值$m$未知, 协方差函数为$R(\tau) = a\e^{-b|\tau|}, \tau \in R, a > 0, b > 0$. 对固定的$T > 0$, 令$\displaystyle \overline{X} = T^{-1}\int^T_0 X(s)\d s$. 证明$\E(\overline{X}) = m$(即$\overline{X}$是$m$的无偏估计)以及
\[\var(\overline{X}) = 2a[(bT)^{-1}-(bT)^{-2}(1-\e^{-bT})].\]
提示:在上述条件下, 期望号与积分号可以交换.
\end{problem}
\begin{solution}
	\[\E(\overline{X}) = \E\left[\frac{1}{T}\int^T_0X(s)\d s\right] = \frac{1}{T}\int^T_0\E[X(s)]\d s = \frac{mT}{T} = m\]
	\[
		\begin{split}
			\var(\overline{X}) & = \E\Bigg[\frac{1}{T^2}\bigg(\int^T_0 X(t)\d t - m\bigg)\bigg(\int^T_0 X(s)\d s - m\bigg)\Bigg]\\
			& = \frac{1}{T^2}\int^T_0\int^T_0 \E\big[(X(t)-m)(X(s)-m)\big]\d s \d t\\
			& = \frac{1}{T^2}\int^T_0\int^T_0 R(t-s)\d s \d t\\
			& = \frac{1}{T^2}\int^T_0\int^T_0 a\e^{-b|t-s|}\d s \d t\\
			& = \frac{2a}{T^2}\int^T_0\d t\int^T_0 \e^{-b|t-s|}\d s\\
			& = \frac{2a}{T^2}\int^T_0\frac{1}{b}\big(1-\e^{-bt}\big)\d t\\
			& = 2a\big[(bT)^{-1}-(bT)^{-2}(1-\e^{-bT})\big].
		\end{split}
	\]
\end{solution}

\begin{problem}{4.13}
设$\{X(t)\}$为平稳过程, 设$\{X(t)\}$的$n$阶导数$X^{(n)}(t)$存在, 证明$\{X^{(n)}(t)\}$是平稳过程.\\
提示:利用协方差函数性质4.
\end{problem}
\begin{solution}
	$\E[X^{(n)}(t)] = \big\{\E[X(t)]\big\}^{(n)} = 0, \cov\big(X^{(n)}(t), X^{(n)}(t+\tau)\big) = (-1)^n R^{(2n)}(\tau)$\\
	$\therefore \{X^{(n)}(t)\}$是平稳过程.
\end{solution}

\begin{problem}{4.14}
证明定理$4.1$中关于平稳序列均值的遍历性定理.\\
提示:用Schwarz不等式
\end{problem}
\begin{solution}
	\textcolor{blue}{充分性}:
	\[
		\begin{split}
			& \quad \E\bigg|\frac{1}{2N+1}\sum^N_{k=-N}X(k)-m\bigg|^2 \qquad (m=\E(X_n))\\
			& = \frac{1}{(2N+1)^2}\E\bigg(\sum^N_{k=-N}X(k)-m\bigg)^2\\
			& = \frac{1}{(2N+1)^2}\E\bigg(\sum^N_{k=-N}X(k)-m\bigg)\bigg(\sum^N_{\ell=-N}X(\ell)-m\bigg)\\
			& = \frac{1}{(2N+1)^2}\sum^N_{k=-N}\,\sum^N_{\ell=-N}R(k-\ell)\\
			& = \frac{1}{(2N+1)^2}\Bigg[\sum^N_{\tau=0}R(\tau)\cdot 2(2N+1-\tau)-(2N+1)R(0)\bigg]\\
			& \leqslant \Bigg|\frac{2}{2N+1}\sum^N_{\tau=0}R(\tau)\Bigg| + \Bigg|\frac{1}{2N+1}R(0)\Bigg|
		\end{split}
	\]
	\[
		\begin{split}
			\because & \lim_{N\to +\infty}\frac{2}{2N+1}\sum^N_{\tau=0}R(\tau) = \lim_{\to +\infty}\frac{2}{2N-1}\sum^{N-1}_{\tau=0}R(\tau) = \lim_{N\to +\infty}\frac{2N}{2N-1}\cdot \frac{1}{N}\sum^{N-1}_{\tau=0}R(\tau) = 0,\\
			& \lim_{N\to +\infty}\frac{1}{2N+1}R(0) = 0,\\
			\therefore & \lim_{N\to +\infty}\E\bigg|\frac{1}{2N+1}\sum^N_{k=-N}X(k)-m\bigg|^2 = 0.
		\end{split}
	\]
	\textcolor{blue}{必要性}:\\
	记$\overline{X}_N = \frac{1}{2N+1}\sum\limits^N_{-N}X_k$, 则有
	\[
		\begin{split}
			& \quad \bigg[\frac{1}{2N+1}\sum^{2N}_{\tau=0}R(\tau)\bigg]^2 = \bigg[\frac{1}{2N+1}\sum^N_{k=-N}\cov(X_{-N}, X_k)\bigg]^2 = \bigg[\cov(X_{-N}, \overline{X}_N)\bigg]^2\\
			& \leqslant \var(X_{-N})\var(\overline{X}_N) \qquad (\text{Schwarz不等式})\\
			& = R(0)\E[(\overline{X}_N-m)^2]\to 0 \quad (N\to +\infty)
		\end{split}
	\]
	从而有
	\[\lim_{N\to +\infty}\frac{1}{2N+1}\sum^{2N}_{\tau=0}R(\tau) = 0,\]
	由上易得
	\[\lim_{N\to +\infty}\frac{1}{N}\sum^{N-1}_{\tau=0}R(\tau) = 0.\]
\end{solution}

\begin{problem}{4.15}
如果$(X_1, X_2, X_3, X_4)$是均值为$0$的联合正态随机向量, 则
\[
	\begin{split}
		\E(X_1 X_2 X_3 X_4) = & \cov(X_1,X_2)\cov(X_3,X_4) + \cov(X_1,X_3)\cov(X_2,X_4)\\
		& + \cov(X_1,X_4)\cov(X_2,X_3).
	\end{split}
\]
利用这个事实证明定理$4.3$
\end{problem}
\begin{solution}
	取固定的$\tau\in \mathbb{Z}$, 记$X_{n+\tau}X_n\overset{\Delta}{=}Y_n$,则
	\[
		\begin{split}
			\E(Y_n) & = R_X(\tau)(const)\\
			\cov(Y_{n+\tau_1}, Y_n) & = \E(Y_{n+\tau_1}Y_n) - R^2_X(\tau)\\
			& = \E(X_{n+\tau_1+\tau}X_{n+\tau_1}X_{n+\tau}X_n) - R^2_X(\tau)\\
			& = R^2_X(\tau) + R^2_X(\tau_1) + R_X(\tau_1+\tau)R_X(\tau_1-\tau) - R^2_X(\tau)\\
			& = R^2_X(\tau_1)+R_X(\tau_1+\tau)R_X(\tau_1-\tau)\\
			& = R_Y(\tau_1)
		\end{split}
	\]
	$\therefore \{Y_n\}$是平稳过程.又易见$X = \{X_n, n \in \mathbb{Z}\}$的协方差函数遍历性成立的充要条件是$Y = \{Y_n, n \in \mathbb{Z}\}$的均值遍历性成立.而我们有
	\[
		\begin{split}
			\Bigg|\frac{1}{N}\sum^{N-1}_{\tau_1=0}R_Y(\tau_1)\Bigg| \leqslant & \frac{1}{N}\sum^{N-1}_{\tau_1=0}\Big|R_Y(\tau_1)\Big|\\
			\leqslant & \frac{1}{N}\sum^{N-1}_{\tau_1=0}\bigg[R^2_X(\tau_1) + \Big(R^2_X(\tau_1+\tau)+R^2_X(\tau_1-\tau)\Big)/2\bigg]\rightarrow 0, (N\rightarrow +\infty)
		\end{split}
	\]
	由均值遍历性定理(i)可知, $Y = \{Y_n, n \in \mathbb{Z}\}$的均值遍历性成立, 即$X = \{X_n, n \in \mathbb{Z}\}$的协方差函数遍历性成立.
\end{solution}

\begin{problem}{4.16}
设$X_0$为随机变量, 其概率密度函数为
\[f(x) =
	\begin{cases}
		2x, & 0 \leqslant x \leqslant 1, \\
		0,  & \text{其他},
	\end{cases}
\]
设$X_{n+1}$在给定$X_0, X_1, \cdots, X_n$下是$(1-X_n,1]$上的均匀分布, $n=0,1,2,\cdots$, 证明$\{X_n, n=0,1,\cdots\}$的均值有遍历性.
\end{problem}
\begin{solution}
	\[\begin{split}
			\E(X_0) & = \int^1_0 2x^2\d x = \frac{2}{3}\\
			\E(X^2_0) & = \int^1_0 2x^3\d x = \frac{1}{2}\\
			\E(X_{n+1}) & = \E[\E(X_{n+1}|X_N)] = \E\Big[\int^1_{1-x_n}\frac{x_{n+1}}{x_n}\d x_{n+1}\Big] = \E(1-\frac{1}{2}X_n)\\
			& = 1-\frac{1}{2}\E(X_n)
		\end{split}
	\]
	$\because \E(X_0) = \frac{2}{3} \quad \therefore \E(X_n) \equiv \frac{1}{2}$
	又有
	\[\E(X^2_{n+1}) = \E\Big[\E(X^2_{n+1}|X_n)\Big] = \E\Big[\int^1_{1-x_n}\frac{x^2_{n+1}}{x_n}\d x_{n+1}\Big] = 1 - \E(X_n) + \frac{1}{3}\E(X^2_n)\]
	$\because \E(X^2_0) = \frac{1}{2} \quad \therefore \E(X^2_n) \equiv \frac{1}{2}$
	\[
		\begin{split}
			\E(X_n X_{n+m}) & = \E\Big[\E(X_n X_{n+m}|X_n)\Big] = \E\Big[X_n\E(X_{n+m}|X_n)\Big]\\
			& = \E\Big[X_n\big(1-\frac{1}{2}\E(X_{n+m-1}|X_n)\big)\Big]\\
			& = \E(X_n) - \frac{1}{2}\E\Big[\E(X_n X_{n+m-1}|X_n)\Big]\\
			& = \frac{2}{3} - \frac{1}{2}\E(X_n X_{n+m-1})
		\end{split}
	\]
	\[\therefore \E(X_n X_{n+m}) - \frac{4}{9} = -\frac{1}{2}\Big(\E(X_n X_{n+m}) - \frac{4}{9}\Big) = \cdots = \Big(-\frac{1}{2}\Big)^m\Big(\E(X^2_n) - \frac{4}{9}\Big) = \frac{1}{18}\Big(-\frac{1}{2}\Big)^m\]
	\[\therefore R_X(n,n+m) = \E\left(X_n - \frac{2}{3}\right)\left(X_{n+m} - \frac{2}{3}\right) = \E(X_n X_{n+m}) - \frac{4}{9}= \frac{1}{18}\Big(-\frac{1}{2}\Big)^m = R(m)\]
	$\therefore \{X_n\}$是平稳序列,又$\because \lim\limits_{m\to +\infty} R(m) = 0 \therefore$是均值遍历的
\end{solution}

\begin{problem}{4.17}
设$\{\varepsilon_n, n=0,\pm1,\cdots\}$为白噪声序列, 令
\[X_n = \alpha X_{n-1} + \varepsilon, |\alpha| < 1, n = \cdots, -1, 0, 1, \cdots,\]
则$X_n = \sum\limits^\infty_{k=0}\alpha^k\varepsilon_{n-k}$, 从而证明$\{X_n, n = \cdots, -1, 0, 1, \cdots\}$为平稳序列. 求出该序列的协方差函数. 此序列是否具有遍历性?
\end{problem}
\begin{solution}
	\[
		\begin{split}
			\E(X_n) & = \sum^{\infty}_{k=0}\alpha^k\E(\varepsilon_{n-k})= 0\\
			R_X(n, n+m) & = \cov(X_n, X_{n+m}) = \E\Big(\sum^{\infty}_{k=0}\alpha^k \varepsilon_{n-k}\Big)\Big(\sum^{\infty}_{\ell=0}\alpha^\ell\varepsilon_{m+n-\ell}\Big)\\
			& = \sum^{\infty}_{k=0}\sum^{\infty}_{\ell=0}\alpha^{k+\ell}\E(\varepsilon_{n-k}\varepsilon_{m+n-\ell}) = \sum^{\infty}_{k=0}\alpha^{2k+m}\E(\varepsilon^2_{n-k})\\
			& = \alpha^m\frac{\sigma^2}{1-\alpha^2}\\
			& = R(m)
		\end{split}
	\]
	$\therefore \{X_n\}$为平稳序列,又$\lim\limits_{m\to +\infty}R(m) = 0,~\therefore$是均值遍历的\\
\end{solution}

\noindent\textcolor{red}{以下没有特殊声明, 所涉及的过程均假定均值函数为$0$}

\begin{problem}{4.18}
我们称一个随机过程$X$为平稳Gauss-Markov过程,如果$X$是平稳Gauss过程,并且具有Markov性,即对任意的$s<t$,任意实数$x_t,x_s,x_u$,有
\[\p(X_t\leqslant x_t|X_s=x_s, X_u=x_u, u<s)=\p(X_t\leqslant x_t|X_s=x_s)\]
试证明零均值的平稳Gauss-Markov过程的协方差函数$R(\tau )$具有$C\e^{-a|\tau|}$这种形式.这里$C$为常数
\end{problem}
\begin{solution}
	本题暂缺
\end{solution}

\begin{problem}{4.19}
设$\{X_n,n=0,1,\cdots \}$是平稳Gauss-Markov序列(即第18题中的$t$取非负整数),均值为0.证明其协方差函数$R(h)$具有$\sigma ^2a^{|h|}$这种形式,其中$|a|\leqslant 1$.
\end{problem}
\begin{solution}
	本题暂缺
\end{solution}

\begin{problem}{4.20}
设$\{X(t)\}$为平稳过程, 令$Y(t) = X(t+a) - X(t-a)$. 分别以$R_X, S_X$和$R_Y, S_Y$记随机过程$X$和$Y$的协方差函数和功率谱密度, 证明
\[
	\begin{split}
		R_Y(\tau) & = 2R_X(\tau) - R_X(\tau + 2a) - R_X(\tau - 2a),\\
		S_Y(\omega) & = 4S_X(\omega)\sin^2a\omega.
	\end{split}
\]
\end{problem}
\begin{solution}
	\[
		\begin{split}
			R(\tau) & = \E\Big[X(t+a)-X(t-a)\Big]\Big[X(t-\tau+a)-X(t-\tau-a)\Big]\\
			& = \E\Big[X(t+a)X(t-\tau-a)\Big] - \E\Big[X(t+a)-X(t-\tau-a)\Big]\\
			& \qquad - \E\Big[X(t-a)X(t-\tau-a)\Big] + \E\Big[X(t-a)-X(t-\tau-a)\Big]\\
			& = R_X(\tau) - R_X(\tau+2a) - R_X(\tau-2a) + R_X(\tau)\\
			& = 2R_X(\tau) - R_X(\tau+2a) - R_X(\tau-2a)\\
			S_Y(\omega) & = \int^{+\infty}_{-\infty}R_Y(\tau)\e^{-i\omega\tau}\d \tau\\
			& = 2\int^{+\infty}_{-\infty}R_X(\tau)\e^{-i\omega\tau}\d \tau - \int^{+\infty}_{-\infty}R_X(\tau+2a)\e^{-i\omega\tau}\d \tau\\
			& \qquad - \int^{+\infty}_{-\infty}R_X(\tau-2a)\e^{-i\omega\tau}\d \tau\\
			& = S_X(\omega)\big(2 - \e^{2a\omega i} - \e^{-2a\omega i}\big)\\
			& = S_X(\omega)\big(2 - 2\cos 2a\omega\big)\\
			& = 4S_X(\omega)\sin^2a\omega
		\end{split}
	\]
\end{solution}

\begin{problem}{4.21}
设平稳过程$X$的协方差函数$R(\tau) = \sigma^2\e^{-\tau^2}$, 试研究其功率谱密度函数的性质.
\end{problem}
\begin{solution}
	\[
		\begin{split}
			S(\omega) & = \int^{+\infty}_{-\infty}\sigma^2\e^{-\tau^2}\e^{-i\omega\tau}\d \tau\\
			& = \sigma^2\e^{-\frac{\omega^2}{4}}\int^{+\infty}_{-\infty}\e^{-(\tau^2+i\omega\tau-\frac{\omega^2}{4})}\d \tau\\
			& = \sigma^2\e^{-\frac{\omega^2}{4}}\int^{+\infty}_{-\infty}\e^{-\frac{(\tau+\frac{i\omega}{2})^2}{2\times \frac{1}{2}}}\d \tau\\
			& = \sigma^2\sqrt{\uppi}\e^{-\frac{\omega^2}{4}}.
		\end{split}
	\]
	$S(\omega)$为$\mathbb{R}$上的实的、偶的、非负且可积的函数.
\end{solution}

\begin{problem}{4.22}
设平稳过程$\{X(t)\}$的协方差函数$R(\tau)=\frac{a^2}{2}\cos \omega \tau +b^2\e^{-a|\tau|}$,求功率谱密度函数$S(\omega )$
\end{problem}
\begin{solution}
	因为$\cos \omega_0 \tau \longleftrightarrow \uppi(\delta(\omega+\omega_0)+\delta(\omega-\omega_0)),\quad \e^{-a|\tau|}\longleftrightarrow \frac{2a}{a^2+\omega ^2}$\\
	故所求谱密度函数为\[S(\omega) = \frac{a^2 \uppi}{2}\big(\delta(\omega+\omega_0)+\delta(\omega-\omega_0)\big)+\frac{2ab^2}{a^2+\omega ^2}\]
\end{solution}

\begin{problem}{4.23}
设$\{X(t)\}$为Gauss平稳过程,均值为零,$R_X(\tau)=A\e^{-a|\tau|}\cos \beta \tau$.令$Y(t)=X^2(t)$,验证$R_Y(\tau)=A^2\e^{-2a|\tau|}(1+\cos 2\beta \tau)$
\end{problem}
\begin{solution}
	由课本4.3.2节中平方检波的结果可知:
	\[\begin{split}
			R_Y(\tau) &= 2R_X^2(\tau) = 2A^2 \e^{-2a |\tau|}\cos^2 \beta\tau = A^2\e^{-2a|\tau|}(1+\cos 2\beta\tau)\\
			&= A^2 (\e^{-2a|\tau|}+\e^{-2a|\tau|}\cos 2\beta\tau)
		\end{split}\]
	由Fourier变换关系:
	\[\e^{-a|\tau|}\longleftrightarrow \frac{2a}{a^2+\omega^2},\quad \e^{-a|\tau|}\cos \omega_0\tau \longleftrightarrow \frac{a}{a^2+(\omega+\omega_0)^2}+\frac{a}{a^2+(\omega-\omega_0)^2}\]
	故可得到$R_Y(\tau)$所对应的谱密度为:
	\[\begin{split}
			S(\omega) &= A^2 \left(\frac{4a}{4a^2+\omega^2}+\frac{2a}{4a^2+(\omega+2\beta)^2}+\frac{2a}{4a^2+(\omega-2\beta)^2}\right)\\
			&= 2aA^2 \left(\frac{2}{4a^2+\omega^2}+\frac{1}{4a^2+(\omega+2\beta)^2}+\frac{1}{4a^2+(\omega-2\beta)^2}\right)
		\end{split}\]
\end{solution}

\begin{problem}{4.24}
设$\{X(t)\}$为Gauss平稳过程,均值为零,功率谱密度$S(\omega)=\frac{1}{1+\omega ^2}$.求$X(t)$落在区间$[0.5,1]$中的概率.
\end{problem}
\begin{solution}
	\[\begin{split}
			R(\tau) & =\frac{1}{2\uppi}\int_{-\infty}^{+\infty}S(\omega)\e^{j\omega \tau}\d \omega = \frac{1}{2\uppi}\int_{-\infty}^{+\infty}\frac{\e^{j\omega \tau}}{1+\omega^2}\d \omega \\
			& =\frac{1}{2\uppi}\cdot 2\uppi j\cdot \res\left[\frac{\e^{jz|\tau|}}{1+z^2},j\right] = \frac{\e^{-|\tau|}}{2}\\
			R(0) & =\var(X(t)) = \E[X^2(t)] = \frac{1}{2} = \sigma ^2
		\end{split}\]
	$\therefore X(t)\sim N(0,\frac{1}{2})\quad \frac{X(t)}{\sigma}=\sqrt{2}X(t)\sim N(0,1)$\\
	$\displaystyle \therefore \p[0.5\leqslant X(t)\leqslant 1] = \p\left[\frac{\sqrt{2}}{2}\leqslant \sqrt{2}X(t)\leqslant \sqrt{2}\right] = \Phi(\sqrt{2}) - \Phi\left(\frac{\sqrt{2}}{2}\right)$
\end{solution}

\begin{problem}{4.25}
已知平稳过程$\{X(t)\}$的功率谱密度为$S(\omega)=\frac{\omega^2+1}{\omega^4 + 4\omega^2 + 3}$,求$X(t)$的均方值.
\end{problem}
\begin{solution}
	\[S(\omega) = \frac{\omega^2}{(\omega^2+1)(\omega^2+3)} = -\frac{1}{2(\omega^2+1)}+\frac{3}{2(\omega^2+3)}\]
	故由上题类似的方法可知$S(\omega)$所对应的$R(\tau) = -\frac{1}{4}\e^{-|\tau|}+\frac{\sqrt{3}}{4}\e^{-\sqrt{3}|\tau|}$,从而求得$X(t)$的均方值:
	\[\E[X^2(t)]=\var(X(t))=R(0)=\frac{\sqrt{3}-1}{4}\qquad (\text{假定}\E[X(t)]=0)\]
\end{solution}

\begin{problem}{4.26}
设$S(\omega)$是功率谱密度函数,证明$\frac{\d^2 S(\omega)}{\d \omega^2}$不可能是功率谱密度函数.
\end{problem}
\begin{solution}[1(郑老师解法)]
	反证法:若$\displaystyle \frac{\d^2 S(\omega)}{\d \omega^2}\xlongequal{\Delta}S''(\omega)$为谱密度函数,则$S''(\omega)\geqslant 0$,从而$S'(\omega)$为单调增.\\
	又$S'(\omega)\leqslant 0$不可能对任何$\omega \in \mathbb{R}$成立(否则$S(\omega)$为单调减),故必存在$\omega_0\in \mathbb{R}$,使得$S'(\omega_0)=a>0$,从而当$\omega \geqslant \omega_0$时,$S'(\omega)\geqslant a$,故而有
	\[\int_{\omega_0}^{\omega}S'(t)\d t \geqslant a(\omega-\omega_0),\quad \text{亦即}S(\omega)\geqslant S(\omega_0)+a(\omega-\omega_0)\]
	这样的$S(\omega)$在$\mathbb{R}$上式不可积的,矛盾
\end{solution}
\begin{solution}[2]
	\[\begin{split}
			\frac{\d^2 S(\omega)}{\d \omega^2} &= \frac{\d^2}{\d \omega^2}\left[\int_{-\infty}^{+\infty}R(\tau)\e^{-j\omega\tau}\d \tau\right] = \int_{-\infty}^{+\infty}-\tau^2 R(\tau)\e^{-j\omega\tau}\d \tau \\
			&= -\int_{-\infty}^{+\infty}\tau^2 R(\tau)\e^{-j\omega\tau}\d \tau
		\end{split}\]
	令$g(\tau)=R(\tau)\e^{-j\omega\tau}$,则存在$\xi \in \mathbb{R},s.t$
	\[
		\begin{split}
			& \int_{-\infty}^{+\infty}\tau^2 g(\tau)\d \tau = \lim_{\tau \to +\infty}\tau^2 \cdot \int_{-\infty}^{\xi}g(\tau)\d \tau + \lim_{\tau \to +\infty}\tau^2 \cdot \int_{\xi}^{+\infty}g(\tau)\d \tau \\
			>\, & 0\cdot \int_{-\infty}^{\xi}g(\tau)\d \tau + 0\cdot \int_{\xi}^{+\infty}g(\tau)\d \tau = 0
		\end{split}
	\]
	$\therefore \frac{\d^2 S(\omega)}{\d \omega^2}<0 \Rightarrow$不可能是功率谱密度函数
\end{solution}

\begin{problem}{4.27}
求下列协方差函数对应的功率谱密度函数:
\begin{enumerate}[label=(\arabic*)]
	\item $R(\tau) = \sigma^2 \e^{-a|\tau|}\cos b\tau$
	\item $R(\tau) = \sigma^2 \e^{-a|\tau|}(\cos b\tau - ab^{-1}\sin b|\tau|)$
	\item $R(\tau) = \sigma^2 \e^{-a|\tau|}(\cos b\tau + ab^{-1}\sin b|\tau|)$
	\item $R(\tau) = \sigma^2 \e^{-a|\tau}(1 + a|\tau| - 2a^2 \tau^2 +a^3|\tau|^3/\varepsilon)$
\end{enumerate}
\end{problem}
\begin{solution}
	\begin{enumerate}[label=(\arabic*)]
		\item $a\sigma^2 \left[\frac{1}{a^2+(\omega+b)^2}+\frac{1}{a^2+(\omega-b)^2}\right]$
		\item $\frac{a\sigma^2 \omega}{b}\left[\frac{1}{a^2+(\omega-b)^2}-\frac{1}{a^2+(\omega+b)^2}\right]$
		\item $a\sigma^2 \left[\frac{2+\frac{\omega}{b}}{a^2+(\omega+b)^2}+\frac{2-\frac{\omega}{b}}{a^2+(\omega-b)^2}\right]$
		\item $\frac{2a\sigma^2}{a^2+\omega^2}+\frac{2a\sigma^2 (a^2-\omega^2)(1-4a^2)}{(a^2+\omega^2)^2}+\frac{4a^3 \sigma^2 (a^4 -4a^2 \omega^2+\omega^4)}{(a^2+\omega^2)^4}$
	\end{enumerate}
\end{solution}

\begin{problem}{4.28}
求下列功率谱密度函数对应的协方差函数:
\begin{enumerate}[label=(\arabic*)]
	\item $\displaystyle S(\omega) = \frac{\omega^2+64}{\omega^4+29\omega^2+100}$
	\item $\displaystyle S(\omega) = \frac{1}{(1+\omega^2)^2}$
	\item $\displaystyle S(\omega) = \sum_{k=1}^{N}\frac{a_k}{\omega^2+b_k^2},N\text{为固定的正整数}$
	\item $\displaystyle S(\omega) =
		      \begin{cases}
			      a, & |\omega|\leqslant b, \\
			      0, & |\omega|>b
		      \end{cases}$
	\item $\displaystyle S(\omega) =
		      \begin{cases}
			      0,   & |\omega|<a\text{或}|\omega|\leqslant 2a, \\
			      b^2, & a\leqslant |\omega|\leqslant 2a.
		      \end{cases}$
\end{enumerate}
\end{problem}
\begin{solution}
	\begin{enumerate}[label=(\arabic*)]
		\item $\frac{5}{7}\e^{-2|\tau|}-\frac{13}{70}\e^{-5|\tau|}$
		\item $\frac{1}{4}\e^{-|\tau|}(1+|\tau|)$
		\item $\sum_{k=1}^{N}\frac{a_k}{2b_k}\e^{-b_k|\tau|}$
		\item $\frac{a\sin b\tau}{\uppi\tau}$
		\item $\frac{b^2}{\uppi\tau}(\sin 2a\tau -\sin a\tau)$
	\end{enumerate}
\end{solution}

\begin{problem}{4.29}
设$\{\varepsilon_n, n=\cdots ,-1,0,1,\cdots \}$为白噪声序列,均值为0,方差为$\sigma^2$.求下列序列的谱密度函数:
\begin{enumerate}[label=(\arabic*)]
	\item $X_n = \varepsilon_n + \alpha_1 \varepsilon_{n-1}, n=\cdots ,-1,0,1,\cdots $
	\item $\displaystyle X_n = \sum_{k=0}^{\infty}\beta_k \varepsilon_{n-k},\text{其中}\sum_{k=0}^{\infty}\beta_k^2<\infty$
\end{enumerate}
\end{problem}
\begin{solution}
	本题暂缺
\end{solution}

\begin{problem}{4.30}
由书中例4.15确定的平稳序列的功率谱密度是周期函数,试作出$(\uppi,\uppi]$中谱密度函数的图形,并讨论当$|\rho|\to 1$时图形如何变化.\\
提示:分$\rho>0$和$\rho<0$讨论.
\end{problem}
\begin{solution}
	本题暂缺
\end{solution}

\begin{problem}{4.31}
设$\{X_n, n=\cdots ,-1,0,1,\cdots \}$为平稳序列,协方差函数为$R(\tau)$,
\begin{enumerate}[label=(\arabic*)]
	\item 求$X_{n+1}$的形如$\hat{X}_{n+1}^{(1)}=aX_n$的最小误差方差预报,这里$a$是待定常数,
	\item 求$X_{n+1}$的形如$\hat{X}_{n+1}^{(2)}=aX_n+bX_{n-1}$的最小均方误差预报,这里$a$和$b$是待定常数.
	\item 上述两个预报$\hat{X}_{n+1}^{(1)}$和$\hat{X}_{n+1}^{(2)}$中,哪个预报的均方误差要小些?试用$R(\tau)$表示它们的差.
	\item 求$X_{n+k}$的形如$\hat{X}_{n+k}aX_n+bX_{n+N}\quad (1\leqslant k\leqslant N)$,的最小均方误差内插,这里$a,b$为待定常数.
	\item 设$\displaystyle Z_n = \sum_{k=0}^{N}X_{n+k}$,其中$N$为固定的正整数.求$Z_n$的形如$\hat{Z}_n = aX_n+bX_{n+N}$的最小均方误差预报,其中$a,b$为待定常数
\end{enumerate}
\end{problem}
\begin{solution}
	\begin{enumerate}[label=(\arabic*)]
		\item 设$\hat{X}^{*}=aX_n$为$X_{n+1}$的最佳预报,则根据投影定理有:
		      \[\E(X_{n+1}-\hat{X}^{*})bX_n = b\E(X_{n+1}X_n-aX_n^2)=0 \quad (\forall b\in \mathbb{R})\]
		      无妨设$b\neq 0$,则有$R(1)-aR(0)=0\Rightarrow a=\frac{R(1)}{R(0)}$
		\item 类似可求出:\[a=\frac{[R(0)-R(2)]R(1)}{R^2(0)-R^(2}(1),\quad b=\frac{[R(0)-R(1)]R(2)}{R^2(0)-R^2(1)}\]
		\item $\hat{X}_{n+1}^{(2)}$的均方误差较小,且有:
		      \[\E[X_{n+1}-\hat{X}_{n+1}^{(2)}]^2-\E[X_{n+1}-\hat{X}_{n+1}^{(1)}]^2 ] \frac{-[R^2(1)-R(0)R(2)]^2}{R^2(0)-R^2(1)}\]
		\item \[a=\frac{R(0)R(k)-R(N-k)R(N)}{R^2(0)-R^2(N)},\quad b=\frac{R(0)R(N-k)-R(k)R(N)}{R^2(0)-R^2(N)}\]
		\item \[a=b=\frac{\sum_{k=0}^{N}R(k)}{R(0)+R(N)}\]
	\end{enumerate}
\end{solution}

\begin{problem}{4.32}
设平稳序列$\{X_n, n=\cdots ,-1,0,1,\cdots \}$的均值为零,协方差函数为$R(h)=\rho ^{|h|},|\rho|<1,h=0,1,\cdots $.求$X_{n+1}$根据$\{X_k, k\leqslant n\}$的线性最佳预报$\hat{X}_{n+1}$.
\end{problem}
\begin{solution}
	本题暂缺\\
\end{solution}

\noindent\textcolor{red}{以下设$\{\varepsilon_n, n=\cdots ,-1,0,1,\cdots \}$是均值为0,,方差为1的白噪声序列}

\begin{problem}{4.33}
证明没有一个平稳序列$\{X_n, n=0,\pm 1,\cdots \}$能满足$X_n=X_{n-1}+\varepsilon_n, n=0,\pm 1,\cdots$.
\end{problem}
\begin{solution}
	本题暂缺
\end{solution}

\begin{problem}{4.34}
证明如下两个滑动平均序列
\[\begin{split}
		X_n & = \varepsilon_n + \alpha \varepsilon_{n-1}\\
		Y_n & = \varepsilon_n + \frac{1}{\alpha}\varepsilon_{n-1}
	\end{split}\]
有相同的自相关函数.
\end{problem}
\begin{solution}
	\begin{enumerate}[label=(\roman*)]
		\item \[\begin{split}
				      \E(X_n X_{n+\tau}) &= \E[(\varepsilon_n+\alpha\varepsilon_{n-1})(\varepsilon_{n+\tau}+\alpha\varepsilon_{n+\tau-1})]\\
				      &=\E(\varepsilon_n \varepsilon _{n+\tau})+\alpha\E(\varepsilon_n \varepsilon_{n+\tau-1})+\alpha^2\E(\varepsilon_{n-1}^2)\\
			      \end{split}\]
		      $\tau=0$时,上式$=\E(\varepsilon_n^2)+2\alpha\E(\varepsilon_n \varepsilon_{n-1}+\alpha^2 \E(\varepsilon_{n-1}^2))=(1+\alpha^2)\sigma^2$\\
		      $\tau=1$时,上式$=0+\alpha\E(\varepsilon_n^2) = \alpha\sigma^2$\\
		      $\tau>1$时,上式$=0$
		      \[r_X(\tau)=\begin{cases}
				      (1+\alpha^2)\sigma^2 \quad & \tau=0 \\
				      \alpha\sigma^2             & \tau=1 \\
				      0                          & \tau>1
			      \end{cases}\]
		\item \[\E(T_n T_{n+\tau}) = \E(\varepsilon_n \varepsilon_{n+\tau})+\frac{1}{\alpha}\E(\varepsilon_n \varepsilon_{n+\tau-1})+\frac{1}{\alpha}\E(\varepsilon_{n-1}\varepsilon_{n+\tau})+\frac{1}{\alpha^2}\E(\varepsilon_{n-1}\varepsilon_{n+\tau-1})\]
		      $\tau=0$时,上式$=(1+\frac{1}{\alpha^2})\sigma^2$\\
		      $\tau=1$时,上式$=\frac{1}{\alpha}\sigma^2$\\
		      $\tau>1$时,上式$=0$
		      \[r_Y(\tau)=\begin{cases}
				      (1+\frac{1}{\alpha^2})\sigma^2 \quad & \tau=0 \\
				      \frac{1}{\alpha}\sigma^2             & \tau=1 \\
				      0                                    & \tau>1
			      \end{cases}\]
	\end{enumerate}
	标准化:
	\[r_X(\tau)=\begin{cases}
			1                         & \tau=0 \\
			\frac{\alpha}{1+\alpha^2} & \tau=1 \\
			0                         & \tau>1
		\end{cases}
		\quad
		r_Y(\tau)=\begin{cases}
			1                         & \tau=0 \\
			\frac{\alpha}{1+\alpha^2} & \tau=1 \\
			0                         & \tau>1
		\end{cases}
	\]
\end{solution}

\begin{problem}{4.35}
设$\{X_n, n=0,\pm 1,\cdots \}$为AR($p$)模型:
\[X_n = \alpha_1 X_{n-1} + \cdots + \alpha_p X_{n-p} + \varepsilon_n,\qquad (n=\cdots ,-1,0,1,\cdots )\]
试导出Yule-Walker方程:
\[R(h) = \alpha_1R(h-1)+\cdots +\alpha_p R(h-p),\qquad h>0\]
提示:AR($p$)模型两边同乘以$X_{n-k}$,然后取期望.
\end{problem}
\begin{solution}
	本题暂缺
\end{solution}

\begin{problem}{4.36}
考虑AR($p$)模型:
\[X_n = \alpha_1 X_{n-1}+\cdots +\alpha_p X_{n-p}+\varepsilon_n,\qquad n=\cdots ,-1,0,1,\cdots \]
假定$1-\alpha Z-\cdots -\alpha_p Z^p$的根都在单位圆外,求功率谱密度函数.
\end{problem}
\begin{solution}
	本题暂缺
\end{solution}

\begin{problem}{4.37}
考虑如下AR(2)模型:
\begin{enumerate}[label=(\arabic*)]
	\item $X_n=0.5X_{n-1}+0.3X_{n-2}+\varepsilon_n$
	\item $X_n=0.5X_{n-1}-0.3X_{n-2}+\varepsilon_n$
\end{enumerate}
试用Yule-Walker方程导出协方差函数,证明它们的谱密度函数$S(\omega)$是周期函数,并作出$S(\omega)$在$(-\uppi,\uppi)$上的图形.
\end{problem}
\begin{solution}
	本题暂缺
\end{solution}

\begin{problem}{4.38}
求下列自回归模型的协方差函数和相关函数:
\begin{enumerate}[label=(\arabic*)]
	\item $X_n=0.8X_{n-1}+\varepsilon_n$
	\item $X_n=0.4X_{n-1}+\varepsilon_n$
	\item $X_n=-0.5X_{n-1}+\varepsilon_n$
\end{enumerate}
\end{problem}
\begin{solution}
	本题暂缺
\end{solution}

\begin{problem}{4.39}
求下列滑动平均模型的协方差函数和相关函数:
\begin{enumerate}[label=(\arabic*)]
	\item $X_n = \varepsilon_n - 0.5\varepsilon _{n-1} - 0.5\varepsilon _{n-2}$
	\item $X_n = \varepsilon_n + 0.6\varepsilon _{n-1} - 0.2\varepsilon _{n-2} - 0.1\varepsilon _{n-3}$
\end{enumerate}
\end{problem}
\begin{solution}
	本题暂缺
\end{solution}

\begin{problem}{4.40}
设$X_n=\varepsilon _n + \beta [\varepsilon _{n-1} + \gamma \varepsilon _{n-2} + \gamma ^2 \varepsilon _{n-3}+\cdots ]$,
这里$\beta$和$\gamma$为常数,$|\gamma|<1$,记$\alpha=\gamma - \beta ,|\alpha|<1$.求$X_{n+1}$根据$\{X_k, k\leqslant n\}$的线性最佳预报.
\end{problem}
\begin{solution}
	本题暂缺
\end{solution}

\begin{problem}{4.41}
考虑AR(2)模型:$X_n=1.8X_{n-1}+0.8X_{n-2}+\varepsilon _n$,求一步预报及$\ell >1$步预报$\hat{X}_{n+1|n},\hat{X}_{n+\ell |n}$.
\end{problem}
\begin{solution}
	本题暂缺
\end{solution}

\begin{problem}{4.42}
考虑AR(2)模型:$X_n=X_{n-1}-0.25X_{n-2}+\varepsilon _n$,求$\hat{X}_{n+\ell |n}$及$\E[(X_{n+\ell}-\hat{X}_{n+\ell |n})^2]$.
\end{problem}
\begin{solution}
	本题暂缺
\end{solution}

\clearpage
\addcontentsline{toc}{section}{参考文献}
\begin{thebibliography}{99}
	\bibitem{jkadbear}jkadbear \emph{方兆本著随机过程第三版 习题答案}\\ Available at \url{https://github.com/jkadbear/Stochastic_Process}
	\bibitem{白鹏}白鹏.\emph{方兆本等著《随机过程》习题解答}[M].
	\bibitem{刘杰班助教}刘杰班助教 \emph{随机过程习题课}
	\bibitem{郑班助教}郑班助教 \emph{随机过程作业习题解答}
	\bibitem{郑坚坚}郑老师 \emph{习题课讲义}
\end{thebibliography}

\appendix
\clearpage
\section{符号说明}
\begin{table}[H]
	\centering
	\begin{tabular}{ccc}
		\toprule
		符号                    & 中文               & 英文                                              \\
		\midrule
		$\E$ $E$              & 期望               & Expectation                                     \\
		$\p$ $P$              & 概率               & Probability                                     \\
		$\bm{P}$              & 状态转移矩阵           & State Transition Matrix                         \\
		$P_{ij}^{(n)}$        & $i$到$j$的$n$步转移概率 & $n$-step transition probability from $i$ to $j$ \\
		$X\sim \Exp(\lambda)$ & 指数分布             & Exponential Distribution                        \\
		$X\sim \poi(\lambda)$ & 泊松分布             & Poisson Distribution                            \\
		\bottomrule
	\end{tabular}
\end{table}