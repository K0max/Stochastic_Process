\section{第一章~引论}
\begin{problem}{1.1}
令$X(t)$为二阶矩存在的随机过程. 试证它是宽平稳的当且仅当$\E[X(s)]$与$\E[X(s)X(s+t)]$都不依赖$s$.
\end{problem}
\begin{solution}
	\textcolor{blue}{充分性:}\\
	若$\E[X(s)]$与$\E[X(s)X(s+s')]$都不依赖$s$\\
	则$\E[X(s)] = $ 常数$m$, $\E[X(s)X(s+s')] = f(t)$\\
	令$s' = s + t$,\\
	\[\therefore \E[X(s)X(s')] = f(s'-s)\]
	\[
		\begin{aligned}
			\therefore R_X(s, s') & = \E[X(s)X(s')] - \E[X(s)]\E[X(s')] \\
			                      & = f(t - s) - m^2
		\end{aligned}
	\]
	$\therefore X(t)$是宽平稳的\\
	\textcolor{blue}{必要性:}\\
	若$X(t)$宽平稳则$\E[X(S)]$为常数$m$, 即$\E[X(S)]$与$s$无关\\ 则
	\[
		\begin{aligned}
			R_X(s, s') & = \E[X(s)X(s')] - \E[X(s)]\E[X(s')] \\
			           & = g(s' - s)                         \\
		\end{aligned}
	\]
	令$s' = s + t$ \\
	则$\E[X(s)X(s+t)] = m^2 + g(t)$与$s$无关\\
\end{solution}

\begin{problem}{1.2}
记$U_1, \cdots, U_n$为在$(0,1)$中均匀分布的独立随机变量. 对$0<t,x<1$定义
\[ I(t,x) =
	\begin{cases}
		1, & \text{ $x \leqslant t$,} \\
		0, & \text{ $x > t$,}
	\end{cases}
\]
并记$\displaystyle X(t) = \frac{1}{n}\sum^n_{k=1}I(t,U_k), 0 \leqslant t \leqslant 1, $这是$U_1, \cdots, U_n$的经验分布函数. 试求过程$X(t)$的均值和协方差函数.
\end{problem}
\begin{solution}
	\[\E[X(t)] = \E\left[\frac{1}{n}\sum^n_{k=1}I(t, U_k)\right] = \E[I(t, U_1)] = \int^t_0 1 \d x = t\]
	\[\begin{aligned}
			R_X(s, t) & = \E [X(s)X(t)] - \E[X(s)]\E[X(t)]                                                                      \\
			          & = \E\left[\frac{1}{n^2}\sum_{i=1}^n I(s, U_i) \cdot \sum_{j=1}^n I(s, U_j)\right] - st                  \\
			          & = \frac{1}{n^2}\Big\{(n^2 - n)\E[I(s, U_1) \cdot I(t, U_2)] + n\E[I(s, U_1) \cdot I(t, U_1)]\Big\} - st \\
			          & = \frac{1}{n^2}[(n^2 - n)st + n \cdot \min(s, t)] - st                                                  \\
			          & = \frac{1}{n}[\min(s, t) - st]
		\end{aligned}\]
\end{solution}

\begin{problem}{1.3}
令$Z_1, Z_2$为独立的正态随机变量,均值为$0$,方差为$\sigma^2$,~$\lambda$为实数.定义过程$X(t) = Z_1\cos\lambda t + Z_2\sin\lambda t$.试求$X(t)$的均值函数和协方差函数.它是宽平稳的吗?
\end{problem}
\begin{solution}
	\[\begin{aligned}
			\E[X(t)]  & = \E(Z_1)\cos \lambda t + \E(Z_2)\sin\lambda t = 0                                      \\
			R_X(s, t) & = \cov[(Z_1\cos\lambda s + Z_2\sin\lambda s), (Z_1\cos\lambda t + Z_2\sin\lambda t)]    \\
			          & = \cov(Z_1, Z_1)\cos\lambda s \cos\lambda t + \cov(Z_2, Z_2)\sin\lambda s \sin\lambda t \\
			          & = \sigma^2\cos\lambda(s - t)                                                            \\
		\end{aligned}\]
	$R_X(s,t)$只与$s-t$有关,故是宽平稳的
\end{solution}

\begin{problem}{1.4}
Poisson过程$X(t), t \geqslant 0 $满足
\begin{enumerate}[label=(\roman*)]
	\item $X(t) = 0$
	\item 对$t>s,\,X(t) - X(s)$~服从均值为$\lambda(t-s)$的Possion分布
	\item 过程是有独立增量的.
\end{enumerate}
试求其均值函数和协方差函数.它是宽平稳的吗?
\end{problem}
\begin{solution}[\color{red}\footnote{注意答案中的协方差函数假设$\color{red} s\geqslant t$}]
	\[\begin{aligned}
			\E[X(t)]  & = \E[X(t)-X(0)] = \lambda t                                           \\
			R_X(s, t) & = \cov[X(t), X(s)]                                                    \\
			          & = \cov\left[\big(X(s)-X(t)+X(t)-X(0)\big), \big(X(t)-X(0)\big)\right] \\
			          & = \cov(X(t) - X(0), X(t) - X(0)) \quad \text{(独立增量)}                  \\
			          & = \lambda t \quad \color{red}(s \geqslant t)
		\end{aligned}\]
	均值不为常数,协方差非仅与$\tau=s-t$有关,故非宽平稳
\end{solution}

\begin{problem}{1.5}
$X(t)$为第4题中的Possion过程. 记$Y(t)=X(t+1)-X(t)$, 试求过程$Y(t)$的均值函数和协方差函数, 并研究其平稳性.
\end{problem}
\begin{solution}
	\[\begin{aligned}
			\E[Y(t)]  & = \E[X(t+1)]-\E[X(t)] = \lambda                                           \\
			R_X(s, t) & = \cov(X(s+1)-X(s), X(t+1)-X(t))                                          \\
			          & = \cov(X(s+1), X(t+1)) + \cov(X(s), X(t))                                 \\
			          & \qquad - \cov(X(s), X(t+1)) - \cov(X(s+1), X(t))                          \\
			          & = \lambda [\min (s+1, t+1) + \min (s, t) - \min (s, t+1) - \min (s+1, t)] \\
		\end{aligned}\]
	令$\beta = s - t$, 当$\beta > 1$或$\beta < -1$时, $R_Y(s, t) = 0$ \\
	当$0 < \beta \leqslant 1$时, $R_Y(s, t) = \lambda (t + 1 + t - s - t) = \lambda (t - s + 1)$\\
	当$-1 \leqslant \beta \leqslant 0$时, $R_Y(s, t) = \lambda (s + 1 + s - s - t) = \lambda (s - t + 1)$\\
	故为宽平稳
\end{solution}

\begin{problem}{1.6}
令$ Z_1 $和$ Z_2 $是独立同分布的随机变量. $\p(Z_1 = -1) = \p(Z_1 = 1) = \frac{1}{2}$. 记$X(t) = Z_1\cos\lambda t + Z_2\sin\lambda t$, $t \in \mathbb{R}$. 试证$X(t)$是宽平稳的, 它是严平稳的吗?
\end{problem}
\begin{solution}
	\[\begin{aligned}
			\E(Z_1)   & = \E(Z_2) = 0                                                                          \\
			\E[X(t)]  & = \E(Z_1)\cos\lambda t + \E(Z_2)\sin\lambda t = 0                                      \\
			R_X(s, t) & = \cov[(Z_1\cos\lambda s + Z_2\sin\lambda s), (Z_1\cos\lambda t + Z_2\sin\lambda t)]   \\
			          & = \cov(Z_1, Z_1)\cos\lambda s \cos\lambda t + \cov(Z_2, Z_2)\sin\lambda s\sin\lambda t \\
			          & = 2 \var(Z_1)\cos\lambda(s-t)                                                          \\
			          & = 2\left[\E (Z_1^2) - \E ^2(Z_1)\right]\cos\lambda(s-t)                                \\
			          & = \cos\lambda(s-t)
		\end{aligned}\]
	故是宽平稳\\
	$F_t(x) = \p(Z_1\cos\lambda t + Z_2\sin\lambda t \leqslant x)$\\
	考虑$F_t(0) = \p(Z_1\cos\lambda t + Z_2\sin\lambda t \leqslant 0)$\\
	当$t = 0$时 $F_t(0) = \p(Z_1 \leqslant 0) = \frac{1}{2}$\\
	当$t = \frac{\uppi}{4\lambda}$时 $F_t(0) = \p\left(\frac{\sqrt{2}}{2}(Z_1+Z_2) \leqslant 0\right) = \frac{3}{4}$\\
	$\therefore F_t(x)$与$t$有关, 故$X(t)$不是严平稳过程
\end{solution}

\begin{problem}{1.7}
试证:若$Z_0, Z_1,\cdots $为独立同分布随机变量, 定义$ X_n = Z_0 + Z_1 + \cdots + Z_n$, 则$\{X_n, n \geqslant 0\}$ 是独立增量过程.
\end{problem}
\begin{solution}
	对$\forall n$及$\forall t_1, \cdots, t_n\in \{0,1,2,\cdots\}, t_1 < t_2 < \cdots < t_n$, 有
	\[\begin{cases}
			X(t_2) - X(t_1) = Z_{t_1+1}+\cdots+Z_{t_2},         \\
			X(t_3) - X(t_2) = Z_{t_2+1}+\cdots+Z_{t_3},         \\
			\quad \vdots                                        \\
			X(t_n) - X(t_{n-1}) = Z_{t_{n-1}+1}+\cdots+Z_{t_n}. \\
		\end{cases}\]
	由题知$Z_{t_1+1}, \cdots, Z_{t_n}$互相独立,\\
	$\therefore(Z_{t_1+1},\cdots,Z_{t_2}),(Z_{t_2+1},\cdots,Z_{t_3}),\cdots,(Z_{t_{n\!-\!1}\!+\!1},\cdots,Z_{t_n})$互相独立,\\
	$\therefore \{X_n, n \geqslant 0\}$为独立增量过程.
\end{solution}

\begin{problem}{1.8}
若$X_1, X_2,\cdots $为独立随机变量, 还要添加什么条件才能确保它是严平稳的随机过程?
\end{problem}
\begin{solution}
	若$\{X_1, X_2, \cdots \}$严平稳, 则对任意正整数$m$和$n$, $X_m$和$X_n$的分布都相同, 从而$X_1, X_2, \cdots $是一列同分布的随机变量. 而当$X_1, X_2, \cdots $是一列独立同分布的随机变量时. 对任意正整数$k$及$n_1, \cdots,n_k, k$维随机向量$\left(X_{n_1}, \cdots, X_{n_k}\right)$的分布函数为(记$X_1, X_2, \cdots $的共同分布函数为$F(x)$)\\
	\[\begin{split}
			& F_{X_{n_1}, \cdots, X_{n_k}}(x_1, \cdots, x_k)  = F_{X_{n_1}}(x_1)\cdots F_{X_{n_k}}(x_n)\\
			= & F(x_1)\cdots F(x_k).\qquad (-\infty < x_1,\cdots,x_k < +\infty)\\
		\end{split}\]
	这说明了$(X_{n_1},\cdots,X_{n_k})$的分布函数与$n_1, \cdots, n_k$无关, 故$\{X_1, X_2, \cdots\}$严平稳.\\
\end{solution}

\begin{problem}{1.9}
令$X$和$Y$是从单位圆内的均匀分布中随机选取一点所得的横坐标和纵坐标. 试计算条件概率
\[\p\left(X^2+Y^2 \geqslant \frac{3}{4} \bigg| X > Y\right).\]
\end{problem}
\begin{solution}
	\begin{minipage}[c]{0.3\textwidth}
		\begin{tikzpicture}[>=Stealth]
			\draw [->] (-1.5,0) -- (1.5,0) node[right] {$x$};
			\draw [->] (0,-1.5) -- (0,1.5) node[above] {$y$};
			\draw (1,0)arc(0:360:1) (0.865,0)arc(0:360:0.865);
			\draw (1,0)node[below]{$(1,0)$};
			\draw (-0.865,0)node[above]{$(-\frac{\sqrt{3}}{2},0)$};
			\fill (1,0)circle(0.05) (-0.865,0)circle(0.05);
			\draw (-1.3,-1.3)--(1.3,1.3)node[above]{$y=x$};
		\end{tikzpicture}
	\end{minipage}
	\begin{minipage}[c]{0.65\textwidth}
		由对称性得,
		\[\begin{aligned}
				  & \p\left(X^2+Y^2 \geqslant \frac{3}{4} \bigg| X > Y\right)      \\
				= & \frac{1}{2}\cdot \p\left(X^2+Y^2 \geqslant \frac{3}{4} \right) \\
				= & \frac{1}{2}\cdot \left(1-\frac{3}{4}\right)=\frac{1}8          \\
			\end{aligned}\]
	\end{minipage}
\end{solution}

\begin{problem}{1.10}
粒子依参数为$\lambda $的Possion分布进入计数器, 两粒子到达的时间间隔$T_1, T_2,\cdots $是独立的参数为$\lambda $的指数分布随机变量. 记$S$是$[0,1]$时段中的粒子总数. 时间区间$I\subset [0,1]$, 其长度记为$|I|$. 试证明$\p(T_1\in I, S = 1) = \p(T_1\in I, T_1 + T_2 > 1)$, 并由此计算$\p(T_1\in I|S = 1) = |I|$.
\end{problem}
\begin{solution}
	设$W_i$为第$i$个离子进入计数器时的时刻,显然有$W_n =\sum_{k=1}^{n}T_k$
	那么有$\{S=1\}=\{W_2>1\}\Rightarrow \{T_1\in I,S=1\}=\{S=1\}=\{T_1\in I,T_1+T_2>1\}$
	于是有\[\p(T_1\in I,S=1)=\p(T_1\in I,T_1+T_2>1)\]
	\[\begin{aligned}
			\p(T_1\in I|S=1) & = \frac{\p(T_1\in I,S=1)}{\p(S=1)}                                                                                   \\
			                 & = \frac{\p[N(t)=0, N(t+|I|=1,N(1)=1)]}{\p[N(1)=1]}                                                                   \\
			                 & = \frac{\e^{-\lambda t}\cdot \lambda |I|\cdot \e^{-\lambda |x|}\cdot \e^{-\lambda (1-t-|I|)}}{\lambda \e^{-\lambda}} \\
			                 & = |I|
		\end{aligned}\]
\end{solution}

\begin{problem}{1.11}
$X, Y$为两独立随机变量且分布相同. 证明$\E(X|X+Y = z) = \E(Y|X+Y = z)$. 并试求基于$X+Y=z$的$X$的最佳预报, 并求出预报误差$\E(X-\phi (X+Y))^2$
\end{problem}
\begin{solution}
	$X,Y$是独立同分布的随机变量,且分布相同,故条件分布相同。故有
	\[\E[X|X+Y=Z] = \E[Y|X+Y=Z]\]
	其于$X+Y=z$的$X$的最佳预报为$\E[X|X+Y=z]=\frac{z}{2}$\\
	同理,预报误差为
	\[\E[X-\varphi(X+Y)]^{2} = \E[\frac{X-Y}{2}]^{2} = \frac{1}{4}[\E(X^{2})-2\E(X)\E(Y)+E(Y^{2}) = \frac{1}{2}\var(X)]\]
\end{solution}

\begin{problem}{1.12}
气体分子的速度$V$有三个垂直分量$V_x, V_y, V_z$, 它们的联合分布密度依Maxwell-Boltzman定律为
\[f_{V_x, V_y, V_z}(v_1, v_2, v_3) = \frac{1}{(2\uppi kT)^{3/2}}\exp\left\{-\left(\frac{v^2_1+v^2_2+v^2_3}{2kT}\right)\right\},\]
其中$k$是Boltzman常数, $T$为绝对温度, 给定分子的总动能为$e$. 试求$x$方向的动量的绝对值的期望值.
\end{problem}
\begin{solution}
	\[\begin{aligned}
			f_{V_x,V_y,V_z}(v_x,v_y,v_z) & = \frac{1}{(2\uppi kT)^{3/2}}\exp\left\{-\frac{v_x^2 + v_y^2 + v_z^2}{2kT}\right\}=\frac{\e^{-\frac{v_x^2}{2kT}}}{\sqrt{2\uppi}\sqrt{kT}}\cdot \frac{\e^{-\frac{v_y^2}{2kT}}}{\sqrt{2\uppi}\sqrt{kT}}\cdot \frac{\e^{-\frac{v_z^2}{2kT}}}{\sqrt{2\uppi}\sqrt{kT}} \\
			V_x,V_y,V_z                  & \sim N(0,kT)                                                                                                                                                                                                                                                      \\
			e                            & = \E\left[\frac{1}{2}mV^2\right]=\frac{1}{2}m\E[V^2]=\frac{1}{2}m\E[V_x^2+V_y^2+V_z]=\frac{1}{2}m\cdot 3kT=\frac{3mkT}{2}                                                                                                                                         \\
			m                            & = \frac{2e}{3kT}                                                                                                                                                                                                                                                  \\
			\E[\left|p_x\right|]         & = m\E[\left|v_x\right|]=m \int_{-\infty}^{+\infty}\frac{1}{\sqrt{2\uppi}\sqrt{kT}}\left|v_x\right|\e^{-\frac{v_x^2}{2kT}}\d v_x                                                                                                                                   \\
			                             & = \frac{2m}{(2\uppi kT)^{\frac{1}{2}}}\int_{0}^{+\infty}v_x \e^{-\frac{v_x^2}{2kT}}\d v_x                                                                                                                                                                         \\
			                             & = m\sqrt{\frac{2kT}{\uppi}} = \frac{2e}{3kT}\sqrt{\frac{2kT}{\uppi}} = \frac{2e}{3}\sqrt{\frac{2}{\uppi kT}}                                                                                                                                                      \\
		\end{aligned}\]
\end{solution}

\begin{problem}{1.13}
若$X_1, X_2,\cdots, X_n$独立同分布. 它们服从参数为$\lambda$的指数分布. 试证$\sum\limits^n_{i=1}X_i$是参数为$(n, \lambda)$的$\Gamma$分布, 其密度为
\[f(t) = \lambda \exp\{-\lambda t\}(\lambda t)^{n-1}/(n-1)!\, ,\quad t \geqslant 0.\]
\end{problem}
\begin{solution}
	$X_i$的矩母函数为\[g_{X_i}(t)=\int_{0}^{+\infty}\lambda \e^{-\lambda x}\cdot \e^{tx}\d x = \frac{\lambda}{\lambda -t}\]
	\[\because X_1,X_2,\dots ,X_n \iid \quad \therefore g_{X_1,X_2,\dots ,X_n}(t) = \left(g_{X_i}(t)\right)^n = \left(\frac{\lambda}{\lambda - t}\right)^n\]
	参数为$(n, \lambda )$的$\Gamma $分布的矩母函数为:
	\[\begin{aligned}
			g_{\Gamma }(t) & = \int_{0}^{+\infty}\frac{\lambda \e^{-\lambda x}(\lambda x)^{n-1}}{(n-1)!}\cdot \e^{tx}\d x                      \\
			               & = \frac{\lambda ^n}{(n-1)!}\int_{0}^{+\infty}x^{n-1}\e^{-(\lambda -t)x}\d x                                       \\
			               & \xlongequal{u=(\lambda -t)x} \frac{\lambda ^n}{(n-1)!}\int_{0}^{+\infty}\frac{u^{n-1}\e^{-u}}{(\lambda -t)^n}\d u \\
			               & = \frac{\lambda ^n}{(n-1)!}\cdot \frac{1}{(\lambda -t)^n}\cdot \Gamma(n)                                          \\
			               & = \frac{\lambda ^n}{(n-1)!}\cdot \frac{1}{(\lambda -t)^n}\cdot (n-1)! = \left(\frac{\lambda}{\lambda -t}\right)^n \\
		\end{aligned}\]
	$\sum_{i=1}^{n}X_i$是参数为$(n,\Gamma )$的$\Gamma $分布
\end{solution}

\begin{problem}{1.14}
设$X_1$和$X_2$为相互独立的均值为${\lambda}_1$和${\lambda}_2$的Possion随机变量. 试求$X_1+X_2$的分布, 并计算给定$X_1+X_2 = n$时$X_1$的条件分布.
\end{problem}
\begin{solution}
	令$Y = X_1+X_2$
	\[\begin{aligned}
			g_Y(t) & = g_{X_1}(t)g_{X_2}(t)                             \\
			       & = \e^{{\lambda}_1(\e^t-1)}\e^{{\lambda}_2(\e^t-1)} \\
			       & = \e^{({\lambda}_1 + {\lambda}_2)(\e^t-1)}
		\end{aligned}\]
	$\therefore Y \sim \poi({\lambda}_1+{\lambda}_2)$\\
	$\therefore $给定$X_1+X_2 = n$时$X_1$服从参数为$p = \frac{{\lambda}_1}{{\lambda}_1 + {\lambda}_2}, n = n$的二项分布
	\[\begin{aligned}
			\p[X_1+X_2=n]       & = \frac{\e^{-(\lambda_1+\lambda_2)}(\lambda_1+\lambda_2)^n}{n!}                                                                                                  \\
			\p[X_1=m|X_1+X_2=n] & = \frac{\p[X_1=m,X_1+X_2=n]}{\p[X_1+X_2=n]}=\frac{\p[X_1=m,X_2=n-m]}{\p[X_1+X_2=n]}                                                                              \\
			                    & = \frac{\frac{\e^{-\lambda_1}\lambda_1^m}{m!}\cdot \frac{\e^{-\lambda_2}\lambda_2^{n-m}}{(n-m)!}}{\frac{\e^{-(\lambda_1+\lambda_2)}(\lambda_1+\lambda_2)^n}{n!}} \\
			                    & = \frac{\lambda_1 ^m \lambda_2 ^{n-m}n!}{m!(n-m)! \lambda_1 ^n \lambda_2 ^n}                                                                                     \\
			                    & = \binom{n}{m}\left(\frac{\lambda_1}{\lambda_1+\lambda_2}\right)^m \left(\frac{\lambda_2}{\lambda_1+\lambda_2}\right)^{n-m}
		\end{aligned}\]
	故服从二项分布
\end{solution}

\begin{problem}{1.15}
若$X_1, X_2,\cdots $独立且有相同的以$\lambda$为参数的指数分布, $N$服从几何分布, 即
\[\p(N = n) = \beta(1-\beta)^{n-1}, n = 1,2,\cdots, 0<\beta<1.\]
试求随机和$\displaystyle Y = \sum_{i=1}^N X_i$的分布.
\end{problem}
\begin{solution}[1]
	\[\E(\e^{tY}|N = n) = g^n_X (t) = \left(\frac{\lambda}{\lambda - t}\right)^n \overset{\Delta}{=} {\alpha}^n\]
	\[\therefore g_Y(t) = \E[\E(\e^{tY}|N)] = \E({\alpha}^N) = \sum^{+\infty}_{n=1}\beta(1-\beta)^{n-1}{\alpha}^n = \sum^{+\infty}_{n=1}\alpha\beta(\alpha-\alpha\beta)^{n-1}\]
	当$|\alpha -\alpha\beta|<1$时$g_Y(t)=\frac{\alpha\beta}{1-\alpha(1-\beta)} = \frac{\lambda\beta}{\lambda\beta - t}$\\
	$\therefore Y$服从参数为$\lambda\beta$的指数分布
\end{solution}
\begin{solution}[2]
	利用题1.13的结论
	\[X_1,\dots ,X_n \iid \quad X_i\sim \Exp(\lambda ) \Rightarrow f(t)=\frac{\lambda \e^{-\lambda t}(\lambda t)^{n-1}}{(n-1)!}\]
	\[\begin{aligned}
			f_Y(y) & = \sum_{n=1}^{\infty}f_{Y|N}(y|n)\p[N=n]=\sum_{n=1}^{\infty}\frac{\lambda ^n\e^{-\lambda t}t^{n-1}}{(n-1)!}\beta(1-\beta)^{n-1} \\
			       & = \beta \e^{-\lambda t}\sum_{n=1}^{\infty}\frac{\lambda ^n\left[(1-\beta)t\right]^{n-1}}{(n-1)!}                                \\
			       & = \lambda \beta \e^{-\lambda t}\sum_{n=1}^{\infty}\frac{(\lambda t(1-\beta ))^{n-1}}{(n-1)!}                                    \\
			       & = \lambda \beta \e^{-\lambda t}\cdot \e^{\lambda t(1-\beta)} = \lambda \beta \e^{-\lambda \beta t}
		\end{aligned}\]
\end{solution}

\begin{problem}{1.16}
若$X_1, X_2, \cdots$独立同分布, $P(X_i = \pm 1) = \frac{1}{2}$. $N$与$X_i$, $i \geqslant 1$独立且服从参数为$\beta$的几何分布, $0 < \beta < 1$. 试求随机和$Y = \sum\limits^N_{i=1} X_i$的均值, 方差和三、四阶矩.
\end{problem}
\begin{solution}[1]
	使用矩母函数法。但据助教的习题提示,没有使用MATLAB等软件,矩母函数法求出来的应该是错误的\cite{郑班助教}
	\[\E(\e^{tY}|N = n) = g^n_X (t) = \E^n(\e^{tY}) = {\left(\frac{\e^t + \e^{-t}}{2}\right)}^n\]
	\[\therefore g_Y(t) = \E\left[\E(\e^{tY}|N)\right] = \E\left[\left(\frac{\e^t + \e^{-t}}{2}\right)^N\,\right] = \sum^{\infty}_{n=1}\left(\frac{\e^t + \e^{-t}}{2}\right)^n \beta (1 - \beta)^{n - 1}\]
	\[\begin{aligned}
			\therefore \E(Y) & = g'_Y(0) = \sum^{\infty}_{n=1} n \left(\frac{\e^t + \e^{-t}}{2}\right)^{n - 1}\beta (1 - \beta)^{n - 1} \frac{\e^t - \e^{-t}}{2} \bigg|_{t=0} = 0                                                                                   \\
			\E(Y^2)          & = g''_Y(0) = \sum^{\infty}_{n=1}\bigg[ n(n-1)\left(\frac{\e^t + \e^{-t}}{2}\right)^{n-2}\beta (1 - \beta)^{n - 1} \left(\frac{\e^t - \e^{-t}}{2}\right)^2 +                                                                          \\
			                 & \qquad n\left(\frac{\e^t + \e^{-t}}{2}\right)^n\beta (1 - \beta)^{n - 1} \bigg] \Bigg|_{t=0} = \sum^{\infty}_{n=1}n\beta(1-\beta)^{n-1} = \frac{1}{\beta}                                                                            \\
			\var(Y)          & = \E(Y^2)-\E^2(Y) = \frac{1}{{\beta}^2}                                                                                                                                                                                              \\
			\E(Y^3)          & = g^{(3)}_Y(0)                                                                                                                                                                                                                       \\
			                 & = \left(\sum^{\infty}_{n=1}\left\{n\beta(1-\beta)^{n-1}\left[(n-1)\left(\frac{\e^t+\e^{-t}}{2}\right)^{n-2}{\frac{\e^t-\e^{-t}}{2}}^2+\left(\frac{\e^t+\e^{-t}}{2}\right)^n\right]\right\}\right)'\Bigg|_{t=0}                       \\
			                 & = \sum^{\infty}_{n=1}\Bigg\{ n\beta{(1-\beta)}^{n-1}\Bigg[(n-1)(n-2)\left(\frac{\e^t+\e^{-t}}{2}\right)^{n-2}\left(\frac{\e^t-\e^{-t}}{2}\right)^{3}+                                                                                \\
			                 & \qquad (n-1)\left(\frac{\e^t+\e^{-t}}{2}\right)^{n-2} \cdot 2 \cdot \frac{\e^t-\e^{-t}}{2} \cdot \frac{\e^t+\e^{-t}}{2} + n \cdot \left(\frac{\e^t+\e^{-t}}{2}\right)^{n-1} \cdot \frac{\e^t-\e^{-t}}{2} \Bigg] \Bigg\} \Bigg|_{t=0} \\
			                 & = 0                                                                                                                                                                                                                                  \\
			\E(Y^4)          & = g^{(4)}_Y(0)                                                                                                                                                                                                                       \\
			                 & = \sum^{\infty}_{n=1}\left\{n\beta(1-\beta)^{n-1}\left[(n-1)\left(\frac{\e^t+\e^{-t}}{2}\right)^{n-1}(\e^t+\e^{-t})+n\left(\frac{\e^t+\e^{-t}}{2}\right)^{n}\right]\right\} \Bigg|_{t=0}                                             \\
			                 & = \sum^{\infty}_{n=1}(3n^2-2n)\beta(1-\beta)^{n-1} = 3\sum^{\infty}_{n=1}n^2\beta(1-\beta)^{n-1} - 2\sum^{+\infty}_{n=1}n\beta(1-\beta)^{n-1}                                                                                        \\
			                 & = 3\left(\frac{1-\beta}{{\beta}^2} + \frac{1}{{\beta}^2}\right) - 2\frac{1}{{\beta}^2} = \frac{6-5\beta}{{\beta}^2}
		\end{aligned}\]
\end{solution}
\begin{solution}[2]
	使用条件期望
	\[\begin{aligned}
			\E(Y|N=n)   & = \E\left(\left.\sum_{i=1}^{N}X_i\right|N=n\right)=\E\left(\sum_{i=1}^{n}X_i\right) = n\E(X_i) =0                                                                       \\
			\E(Y)       & = \sum_{i=1}^{\infty}\E(Y|N=n)\p(N=n) = 0                                                                                                                               \\
			\E(Y^2|N=n) & = \E\left[\left.\left(\sum_{i=1}^{N}X_i\right)^2\right|N=n\right] = \E\left(\sum_{i=1}^{n}X_i^2+2\sum_{1\leqslant i<j\leqslant n}X_i X_j\right)                         \\
			            & = n\E(X_i^2)+2\sum \E(X_i)\E(X_j) = n                                                                                                                                   \\
			\E(Y^2)     & = \sum_{i=1}^{\infty}\E(Y^2|N=n)\p(N=n) = \sum_{i=1}^{\infty}n\p(N=n) = \sum_{i=1}^{\infty}n\beta (1-\beta )^{n-1} = \frac{1}{\beta}                                    \\
			\E(Y^3|N=n) & = \E\left(\sum_{i=1}^{n}X_i^3+3\sum_{1\leqslant i<j\leqslant n}X_i X_j^2+3\sum_{1\leqslant i<j\leqslant n}X_i^2 X_j\right)                                              \\
			            & = \E\left(\sum_{i=1}^{n}X_i^3 +3\sum_{i\neq j}X_i^2 X_j\right) = n\E(X_i^3) +3\sum_{i\neq j}\E(X_i^2)\E(X_j) = 0                                                        \\
			\E(Y^3)     & = \sum_{i=1}^{\infty}\E(Y^3|N=n)\p(N=n) = 0                                                                                                                             \\
			\E(Y^4|N=n) & = \E\left(\sum_{i=1}^{n}X_i^4+4\sum_{1\leqslant i<j\leqslant n}X_i X_j^3+6\sum_{1\leqslant i<j\leqslant n}X_i^2 X_j^2+4\sum_{1\leqslant i<j\leqslant n}X_i^3 X_j\right) \\
			            & = \E\left(\sum_{i=1}^{n}X_i^4 + 4\sum_{i\neq j}X_i^3 X_j + 6\sum_{i\neq j}X_i^2 X_j^2\right)                                                                            \\
			            & = n\E(X_i^4) + 0 + 6\cdot \binom{n}{2}\E(X_i^2 X_j^2) = n + 6\cdot \binom{n}{2} = 3n^2 - 2n                                                                             \\
			\E(Y^4)     & = \sum_{i=1}^{\infty}\E(Y^4|N=n)\p(N=n) = \sum_{i=1}^{\infty}3n^2\beta (1-\beta )^{n-1} - 2\sum_{i=1}^{\infty}n\beta (1-\beta )^{n-1}                                   \\
			            & = \frac{6-5\beta}{{\beta}^2}
		\end{aligned}\]
	计算$\E(Y^4)$时会出现一项$\sum_{i=1}^{\infty}n^2\beta (1-\beta )^{n-1}$,计算方法如下:
	\[\begin{aligned}
			\sum_{i=1}^{\infty}n^2\beta (1-\beta )^{n-1} & = \beta \sum_{i=1}^{\infty}[n(n-1)+n](1-\beta )^{n-1}                                                                        \\
			                                             & = \beta \left[\sum_{i=1}^{\infty}n(n-1)(1-\beta )^{n-1} + \sum_{i=1}^{\infty}n(1-\beta)^{n-1}\right]                         \\
			                                             & = \beta \left[(1-\beta )\sum_{n=1}^{\infty}\left((1-\beta )^n\right)'' + \sum_{n=1}^{\infty}\left((1-\beta)^n\right)'\right] \\
			                                             & = \beta \left[(1-\beta)\left(\frac{1}{\beta}-1\right)'' - \left(\frac{1}{\beta} - 1\right)'\right]                           \\
			                                             & = \beta \left[\frac{2\beta(1-\beta)}{\beta ^3} + \frac{1}{\beta ^2}\right] = \frac{2-\beta}{\beta ^2}
		\end{aligned}\]
	或者可以采用稍微有点数学性的方法:
	$\E(Y^4) = \beta \sum_{n=1}^{\infty}n^2 (1-\beta)^{n-1}$在$(0,1)$上内闭一致收敛,$\therefore$求和和偏导次序可交换

	令$a = 1-\beta \Rightarrow n^2 a^{n-1}=\frac{\partial }{\partial a}(a\cdot \frac{\partial }{\partial a}a^n)$
	\[\begin{aligned}
			\sum_{n=1}^{\infty}n^2 a^{n-1} & = \sum_{n=1}^{\infty}\frac{\partial }{\partial a}\left(a \frac{\partial a^n}{\partial a}\right) = \frac{\partial }{\partial a}\left(a \frac{\partial \sum_{n=1}^{\infty}a^n}{\partial a}\right) \\
			                               & = \frac{\partial }{\partial a}\left(a \frac{\partial (\frac{a}{1-a})}{\partial a}\right) = \frac{1+a}{(1-a)^3} = \frac{2-\beta}{\beta ^3}
		\end{aligned}\]
\end{solution}

\begin{problem}{1.17}
随机变量$N$服从参数为$\lambda$的Possion分布. 给定$N=n$, 随机变量$M$服从以$n$和$p$为参数的二项分布. 试求$M$的无条件概率分布.
\end{problem}
\begin{solution}
	\[
		\begin{split}
			& \E\big(\e^{tM}|N=n\big) = \big(p\e^t+(1-p)\big)^n \overset{\Delta}{=} a^n\\
			& g_M(t) = \E(a^N) = \sum^{+\infty}_{n=0}a^n\frac{\lambda^n \e^{-\lambda}}{n!} = \e^{\lambda p(\e^t-1)}\\
			& \therefore M \sim \poi(\lambda p)
		\end{split}
	\]
\end{solution}